\usefonttheme[onlymath]{serif}
\usepackage[utf8]{inputenc}
\usepackage{epsfig}
\usepackage{amsmath}
\usepackage{amssymb}
\usepackage{amsfonts}
\usepackage{pifont}
\usepackage{cancel}
\usepackage[normalem]{ulem}
\usepackage{xcolor}
\usepackage{cases}
\usepackage{comment}
\usepackage{mathpazo}
\usepackage{rotating}
\usepackage{ifthen}
\usepackage{times}
% \usepackage{transparent}
\usepackage{xstring}
\usepackage[absolute,overlay]{textpos} %showboxes for debug
\usepackage{multimedia}  %for movies
\usepackage{media9}
%%% for notes
\usepackage{pgfpages}
\usepackage{tcolorbox}
% to zoom with tikz%%%%%%%%%%%%
\usepackage{tikz}
\usetikzlibrary[patterns]
\usetikzlibrary{arrows,shapes,backgrounds,calc,decorations,spy}
\usetikzlibrary{positioning}
%%%%%%%%%%%%%%%%%%%%%%%%%%%%%%%
\usepackage{appendixnumberbeamer}
\usepackage{fancybox}
\setlength{\TPHorizModule}{\paperwidth}\setlength{\TPVertModule}{\paperheight}% for textpos
\usepackage{mathabx} % for astronomical symbols of planets

% checkmark and crossmark
% requires pifont
\newcommand{\cmark}{\ding{51}}%
\newcommand{\xmark}{\ding{55}}%

% To draw arrows between given points
\newcommand\tikzmark[1]{%
  \tikz[remember picture,overlay]\node (#1) {};%
}

\newcommand\Connect[3][]{%
  \tikz[remember picture,overlay]
  \draw[-to,line width=1pt,>=latex,#1] (#2.north east) -- ( $ (#3.north west) + (-20pt,0) $ );%
}

\newcommand\Connectw[3][]{%
  \tikz[remember picture,overlay]
  \draw[-to,line width=3pt,>=latex,white!80!Yellow,#1] (#2.east) -- ( $ (#3.west) + (-2pt,0) $ );%
}

\newcommand\Connectr[3][]{%
  \tikz[remember picture,overlay]
  \draw[-to,line width=3pt,>=latex,red!80!Blue,#1] (#2.east) -- ( $ (#3.east) + (-2pt,0) $ );%
}

\newcommand\Connectlr[3][]{%
  \tikz[remember picture,overlay]
  \draw[-to,line width=3pt,>=latex,red,#1] (#2.east) -- ( $ (#3.east) + (-2pt,0) $ );%
}

\newcommand\Connectb[3][]{%
  \tikz[remember picture,overlay]
  \draw[-to,line width=5pt,>=latex,Blue,#1] (#2.east) -- ( $ (#3.east) + (-2pt,0) $ );%
}

\newcommand\ConnectCol[4][]{%
  \tikz[remember picture,overlay]
  \draw[-to,line width=4pt,>=latex,#1,#2] (#3.east) -- ( $ (#4.east) + (-2pt,0) $ );%
}


\newcommand{\hcancel}[1]{%
  \tikz[baseline=(tocancel.base)]{
    \node[inner sep=0pt,outer sep=0pt] (tocancel) {#1};
    \draw[red] (tocancel.south west) -- (tocancel.north east);
  }%
}%

\newcommand{\vdhdiagram}[9]{%Credits: Ruben Boots 2016
  \tikzset{
    cheating dash/.code args={on #1 off #2}{
      % Use csname so catcode of @ doesn't have do be changed.
      \csname tikz@addoption\endcsname{%
        \pgfgetpath\currentpath%
        \pgfprocessround{\currentpath}{\currentpath}%
        \csname pgf@decorate@parsesoftpath\endcsname{\currentpath}{\currentpath}%
        \pgfmathparse{\csname pgf@decorate@totalpathlength\endcsname}\let\rest=\pgfmathresult%
        \pgfmathparse{#1+#2}\let\onoff=\pgfmathresult%
        \pgfmathparse{max(floor(\rest/\onoff), 1)}\let\nfullonoff=\pgfmathresult%
        \pgfmathparse{max((\rest-\onoff*\nfullonoff)/\nfullonoff+#2, #2)}\let\offexpand=\pgfmathresult%
        \pgfsetdash{{#1}{\offexpand}}{0pt}}%
    }
  }
  \newboolean{RLOFone}
  \newboolean{RLOFtwo}
  \newboolean{CE}

  \def\mid{9}             % Horizontal position middle of system column

  \foreach
  \aone /\atwo /\xrone /\xrtwo /\xRLOFone /\xRLOFtwo /\xCE
  in {#1 / #2  / #3  / #4   / #5   / #6  / #7}
  {
    \pgfmathsetmacro{\yprev}{0} %dummy initialization
    \pgfmathsetmacro{\a}{max(\aone,\atwo)}  % Gets biggest RL-size for correct vertical placement
    \def\rone{\xrone*\aone}                 % Radius first star in terms of its RL size
    \def\rtwo{\xrtwo*\atwo}                 % Radius second star in terms of its RL size
    \def\y{\yprev-\a-.3 }                   % Vertical position system.

    \ifthenelse{\xRLOFone = 1}
    {\setboolean{RLOFone}{true}}
    {\setboolean{RLOFone}{false}}
    \ifthenelse{\xRLOFtwo = 1}
    {\setboolean{RLOFtwo}{true}}
    {\setboolean{RLOFtwo}{false}}
    \ifthenelse{\xCE = 1}
    {\setboolean{CE}{true}}
    {\setboolean{CE}{false}}

    \draw[dashed, line width=1pt]

    (\mid-\aone-0.2,\y)++(45:\aone) arc (45:315:\aone)
    (\mid+\atwo+0.2,\y)++(225:\atwo) arc (-135:135:\atwo)
    (\mid-\aone-0.2,\y)++(45:\aone) to[out=-45,in=135] ($ ({\mid+\atwo+0.2-\atwo/sqrt(2)},{\y-\atwo/sqrt(2)}) $)
    (\mid-\aone-0.2,\y)++(-45:\aone) to[out=45,in=-135] ($ ({\mid+\atwo+0.2-\atwo/sqrt(2)},{\y+\atwo/sqrt(2)}) $);


    \ifthenelse{\boolean{RLOFone}}
    {
      \draw[fill, color=#8]
      (\mid-\aone-0.2,\y) circle (\aone);
      \draw[fill, color=#9]
      (\mid+\atwo+0.2,\y) circle (\rtwo);
      \draw[fill, color=#8, opacity=0.5, draw opacity=0]
      (\mid-\aone-0.2,\y)++(45:\aone) to[out=-45,in=180] ($ ({\mid+\atwo+0.2},\y-\rtwo) $)
      to[out=180,in=45] ($ ({\mid-\aone-0.2+\aone/sqrt(2)},{\y-\aone/sqrt(2)}) $);
    }
    {
      \ifthenelse{\boolean{RLOFtwo}}
      {
        \draw[fill, color=#8]
        (\mid-\aone-0.2,\y) circle (\rone);
        \draw[fill, color=#9]
        (\mid+\atwo+0.2,\y) circle (\atwo);
        \draw[fill, color=#9, opacity=0.5, draw opacity=0]
        (\mid+\atwo+0.2,\y)++(135:\atwo) to[out=225,in=0] ($ ({\mid-\aone-0.2},\y-\rone) $)
        to[out=0,in=135] ($ ({\mid+\atwo+0.2-\atwo/sqrt(2)},{\y-\atwo/sqrt(2)}) $);
      }
      {
        \ifthenelse{\boolean{CE}}
        {
          \ifthenelse{\lengthtest{\aone pt > \atwo pt}}
          {
            \draw[fill, color=#8, opacity=0.5, draw opacity =0]
            (\mid-\aone-0.2,\y) circle (\aone+0.1)
            (\mid+\atwo+0.2,\y) circle (\atwo+0.1)
            (\mid+\atwo+0.2,\y+\atwo+0.1) to[out=180,in=-30] ($ ({\mid-\aone-0.2+(\aone+0.1)/2},{\y+(\aone+0.1)*sqrt(3)/2}) $)--
            ($ ({\mid-\aone-0.2+(\aone+0.1)/2},{\y-(\aone+0.1)*sqrt(3)/2}) $) to[out=30,in=180] ($ ({\mid+\atwo+0.2},{\y-\atwo-0.1}) $);
            \draw[fill, color=#8, overlay]
            (\mid-\aone-0.2,\y) circle (\rone)
            (\mid+\atwo+0.2,\y) circle (\rtwo);
          }
          {
            \ifthenelse{\lengthtest{\aone pt < \atwo pt}}
            {
              \draw[fill, color=#8, opacity=0.5, draw opacity =0]
              (\mid-\aone-0.2,\y) circle (\aone+0.1)
              (\mid+\atwo+0.2,\y) circle (\atwo+0.1)
              (\mid+\atwo+0.2,\y)++(120:\atwo+0.1) to[out=210,in=0] ($ ({\mid-\aone-0.2},{\y+(\aone+0.1)}) $)--
              ($ ({\mid-\aone-0.2},{\y-(\aone+0.1)}) $) to[out=0,in=150] ($ ({\mid+\atwo+0.2-(\atwo+0.1)/2},{\y-(\atwo+0.1)*sqrt(3)/2}) $);
              \draw[fill, color=#8]
              (\mid-\aone-0.2,\y) circle (\rone)
              (\mid+\atwo+0.2,\y) circle (\rtwo);
            }
            {
              \draw[fill, color=#8, opacity=0.5, draw opacity =0]
              (\mid-\aone-0.2,\y) circle (\aone+0.1)
              (\mid+\atwo+0.2,\y) circle (\atwo+0.1)
              (\mid+\atwo+0.2,\y)++(120:\atwo+0.1) to[out=210,in=-30] ($ ({\mid-\aone-0.2+(\aone+0.1)/2},{\y+(\aone+0.1)*sqrt(3)/2}) $)--
              ($ ({\mid-\aone-0.2+(\aone+0.1)/2},{\y-(\aone+0.1)*sqrt(3)/2}) $) to[out=30,in=150] ($ ({\mid+\atwo+0.2-(\atwo+0.1)/2},{\y-(\atwo+0.1)*sqrt(3)/2}) $);
              \draw[fill, color=#8]
              (\mid-\aone-0.2,\y) circle (\rone)
              (\mid+\atwo+0.2,\y) circle (\rtwo);
            }
          }
        }
      }
      {
        \draw[fill, color=#8]
        (\mid-\aone-0.2,\y) circle (\rone);
        \draw[fill, color=#9]
        (\mid+\atwo+0.2,\y) circle (\rtwo);
      }
    }


    \pgfmathparse{\y-\a}
    \global\let\yprev\pgfmathresult
  }
}

\newcommand{\vdhdiagramSN}[9]{%Credits: Ruben Boots 2016
  \tikzset{
    cheating dash/.code args={on #1 off #2}{
      % Use csname so catcode of @ doesn't have do be changed.
      \csname tikz@addoption\endcsname{%
        \pgfgetpath\currentpath%
        \pgfprocessround{\currentpath}{\currentpath}%
        \csname pgf@decorate@parsesoftpath\endcsname{\currentpath}{\currentpath}%
        \pgfmathparse{\csname pgf@decorate@totalpathlength\endcsname}\let\rest=\pgfmathresult%
        \pgfmathparse{#1+#2}\let\onoff=\pgfmathresult%
        \pgfmathparse{max(floor(\rest/\onoff), 1)}\let\nfullonoff=\pgfmathresult%
        \pgfmathparse{max((\rest-\onoff*\nfullonoff)/\nfullonoff+#2, #2)}\let\offexpand=\pgfmathresult%
        \pgfsetdash{{#1}{\offexpand}}{0pt}}%
    }
  }
  \newboolean{RLOFone}
  \newboolean{RLOFtwo}
  \newboolean{CE}

  \def\mid{9}             % Horizontal position middle of system column

  \foreach
  \aone /\atwo /\xrone /\xrtwo /\xRLOFone /\xRLOFtwo /\xCE
  in {#1 / #2  / #3  / #4   / #5   / #6  / #7}
  {
    \pgfmathsetmacro{\yprev}{0} %dummy initialization
    \pgfmathsetmacro{\a}{max(\aone,\atwo)}  % Gets biggest RL-size for correct vertical placement
    \def\rone{\xrone*\aone}                 % Radius first star in terms of its RL size
    \def\rtwo{\xrtwo*\atwo}                 % Radius second star in terms of its RL size
    \def\y{\yprev-\a-.3 }                   % Vertical position system.

    \ifthenelse{\xRLOFone = 1}
    {\setboolean{RLOFone}{true}}
    {\setboolean{RLOFone}{false}}
    \ifthenelse{\xRLOFtwo = 1}
    {\setboolean{RLOFtwo}{true}}
    {\setboolean{RLOFtwo}{false}}
    \ifthenelse{\xCE = 1}
    {\setboolean{CE}{true}}
    {\setboolean{CE}{false}}

    \ifthenelse{\boolean{RLOFone}}
    {
      \draw[fill, color=#8]
      (\mid-\aone-0.2,\y) circle (\aone);
      \draw[fill, color=#9]
      (\mid+\atwo+0.2,\y) circle (\rtwo);
      \draw[fill, color=#8, opacity=0.5, draw opacity=0]
      (\mid-\aone-0.2,\y)++(45:\aone) to[out=-45,in=180] ($ ({\mid+\atwo+0.2},\y-\rtwo) $)
      to[out=180,in=45] ($ ({\mid-\aone-0.2+\aone/sqrt(2)},{\y-\aone/sqrt(2)}) $);
    }
    {
      \ifthenelse{\boolean{RLOFtwo}}
      {
        \draw[fill, color=#8]
        (\mid-\aone-0.2,\y) circle (\rone);
        \draw[fill, color=#9]
        (\mid+\atwo+0.2,\y) circle (\atwo);
        \draw[fill, color=#9, opacity=0.5, draw opacity=0]
        (\mid+\atwo+0.2,\y)++(135:\atwo) to[out=225,in=0] ($ ({\mid-\aone-0.2},\y-\rone) $)
        to[out=0,in=135] ($ ({\mid+\atwo+0.2-\atwo/sqrt(2)},{\y-\atwo/sqrt(2)}) $);
      }
      {
        \ifthenelse{\boolean{CE}}
        {
          \ifthenelse{\lengthtest{\aone pt > \atwo pt}}
          {
            \draw[fill, color=#8, opacity=0.5, draw opacity =0]
            (\mid-\aone-0.2,\y) circle (\aone+0.1)
            (\mid+\atwo+0.2,\y) circle (\atwo+0.1)
            (\mid+\atwo+0.2,\y+\atwo+0.1) to[out=180,in=-30] ($ ({\mid-\aone-0.2+(\aone+0.1)/2},{\y+(\aone+0.1)*sqrt(3)/2}) $)--
            ($ ({\mid-\aone-0.2+(\aone+0.1)/2},{\y-(\aone+0.1)*sqrt(3)/2}) $) to[out=30,in=180] ($ ({\mid+\atwo+0.2},{\y-\atwo-0.1}) $);
            \draw[fill, color=#8, overlay]
            (\mid-\aone-0.2,\y) circle (\rone)
            (\mid+\atwo+0.2,\y) circle (\rtwo);
          }
          {
            \ifthenelse{\lengthtest{\aone pt < \atwo pt}}
            {
              \draw[fill, color=#8, opacity=0.5, draw opacity =0]
              (\mid-\aone-0.2,\y) circle (\aone+0.1)
              (\mid+\atwo+0.2,\y) circle (\atwo+0.1)
              (\mid+\atwo+0.2,\y)++(120:\atwo+0.1) to[out=210,in=0] ($ ({\mid-\aone-0.2},{\y+(\aone+0.1)}) $)--
              ($ ({\mid-\aone-0.2},{\y-(\aone+0.1)}) $) to[out=0,in=150] ($ ({\mid+\atwo+0.2-(\atwo+0.1)/2},{\y-(\atwo+0.1)*sqrt(3)/2}) $);
              \draw[fill, color=#8]
              (\mid-\aone-0.2,\y) circle (\rone)
              (\mid+\atwo+0.2,\y) circle (\rtwo);
            }
            {
              \draw[fill, color=#8, opacity=0.5, draw opacity =0]
              (\mid-\aone-0.2,\y) circle (\aone+0.1)
              (\mid+\atwo+0.2,\y) circle (\atwo+0.1)
              (\mid+\atwo+0.2,\y)++(120:\atwo+0.1) to[out=210,in=-30] ($ ({\mid-\aone-0.2+(\aone+0.1)/2},{\y+(\aone+0.1)*sqrt(3)/2}) $)--
              ($ ({\mid-\aone-0.2+(\aone+0.1)/2},{\y-(\aone+0.1)*sqrt(3)/2}) $) to[out=30,in=150] ($ ({\mid+\atwo+0.2-(\atwo+0.1)/2},{\y-(\atwo+0.1)*sqrt(3)/2}) $);
              \draw[fill, color=#8]
              (\mid-\aone-0.2,\y) circle (\rone)
              (\mid+\atwo+0.2,\y) circle (\rtwo);
            }
          }
        }
      }
      {

        \draw[fill, color=#8]
        (\mid-\aone-0.2,\y) circle (\rone);
        \draw[color=#8,
        pattern=crosshatch, pattern
        color=red]
        (\mid-\aone-0.2,\y) circle (\rone);
        \draw[fill, color=#9]
        (\mid+\atwo+0.2,\y) circle (\rtwo);
      }
    }

    \draw[dashed, line width=1pt]

    (\mid-\aone-0.2,\y)++(45:\aone) arc (45:315:\aone)
    (\mid+\atwo+0.2,\y)++(225:\atwo) arc (-135:135:\atwo)
    (\mid-\aone-0.2,\y)++(45:\aone) to[out=-45,in=135] ($ ({\mid+\atwo+0.2-\atwo/sqrt(2)},{\y-\atwo/sqrt(2)}) $)
    (\mid-\aone-0.2,\y)++(-45:\aone) to[out=45,in=-135] ($ ({\mid+\atwo+0.2-\atwo/sqrt(2)},{\y+\atwo/sqrt(2)}) $);

    \pgfmathparse{\y-\a}
    \global\let\yprev\pgfmathresult
  }
}

\newcommand{\udef}{\stackrel{\mathrm{def}}{=}}

\newcommand{\todo}[1]{{\Huge $\blacksquare$~\textbf{\color{red}#1}~$\blacksquare$}\\ }

\newenvironment<>{varblock}[2][.9\textwidth]{%
  \setlength{\textwidth}{#1}
  \begin{actionenv}#3%http://www.latex-community.org/forum/viewtopic.php?f=4&t=4203
    \def\insertblocktitle{#2}%
    \par%
    \usebeamertemplate{block begin}}
  {\par%
    \usebeamertemplate{block end}%
  \end{actionenv}}

% some colors
\definecolor{bblue}{HTML}{144b7a}%unipi logo color
\definecolor{blue_mesa}{rgb}{58,86,253}%mesa logo color
\definecolor{gray}{gray}{0.}
\definecolor{Yellow}{HTML}{FFFF99}
\definecolor{Green}{rgb}{0,0.6,0}
\definecolor{Orange}{HTML}{FF9900}
\definecolor{Mauve}{rgb}{0.58,0,0.82}
\definecolor{lightred}{HTML}{FF3333}
\definecolor{redinplot}{HTML}{FF0000}
\definecolor{yellowinplot}{HTML}{BFBF00}
\definecolor{cyaninplot}{HTML}{00BFBF}
\definecolor{reddish}{HTML}{FF8484}
\definecolor{yellowish}{HTML}{FFF684}
\definecolor{pinkish}{HTML}{FF94FF}
\colorlet{whiteish}{white!80!Yellow}
\colorlet{lightgray}{gray!50}

%% set theme
\usetheme{metropolis}   %%% Theme Name
% customizations
\metroset{progressbar=frametitle}
\metroset{subsectionpage=progressbar}
\setbeamertemplate{blocks}[rounded]
% colors
\usecolortheme{seahorse}
\setbeamercolor{structure}{fg=Blue, bg=white}
\setbeamercolor{progress bar}{fg=Blue, bg=gray!30}
\setbeamercolor{progress bar in section page}{fg=Blue, bg=gray!30}
\setbeamercolor{frametitle}{bg=, fg=Blue} %% set colors

%block
\setbeamercolor{block body}{use=structure,bg=white!80!Yellow}
\setbeamercolor{block title}{use=Blue,fg=Blue,bg=white!80!Yellow}

% \setbeamercolor{block body}{bg=white,fg=black} %% white background
% \setbeamercolor{block title}{bg=white,fg=Blue} %% white background
\setbeamercolor{background canvas}{bg=,fg=black} %% white background
% remove margins
\setbeamersize{text margin left=2pt,text margin right=2pt}
\setbeamertemplate{section in toc}{
  \centering \bf \color{Blue}\inserttocsection \par}
\setbeamertemplate{subsection in toc}{
  \centering \inserttocsubsection \par}

%%% Local Variables:
%%% mode: plain-tex
%%% TeX-master: t
%%% End:
