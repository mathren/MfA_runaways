%%% Author: Mathieu Renzo
\documentclass[xcolor=dvipsnames,professionalfonts, aspectratio=169]{beamer}
\usefonttheme[onlymath]{serif}
\usepackage[utf8]{inputenc}
\usepackage{epsfig}
\usepackage{amsmath}
\usepackage{amssymb}
\usepackage{amsfonts}
\usepackage{pifont}
\usepackage{cancel}
\usepackage[normalem]{ulem}
\usepackage{xcolor}
\usepackage{cases}
\usepackage{comment}
\usepackage{mathpazo}
\usepackage{rotating}
\usepackage{ifthen}
\usepackage{times}
% \usepackage{transparent}
\usepackage{xstring}
\usepackage[absolute,overlay]{textpos} %showboxes for debug
\usepackage{multimedia}  %for movies
\usepackage{media9}
%%% for notes
\usepackage{pgfpages}
\usepackage{tcolorbox}
% to zoom with tikz%%%%%%%%%%%%
\usepackage{tikz}
\usetikzlibrary[patterns]
\usetikzlibrary{arrows,shapes,backgrounds,calc,decorations,spy}
\usetikzlibrary{positioning}
%%%%%%%%%%%%%%%%%%%%%%%%%%%%%%%
\usepackage{appendixnumberbeamer}
\usepackage{fancybox}
\setlength{\TPHorizModule}{\paperwidth}\setlength{\TPVertModule}{\paperheight}% for textpos
\usepackage{mathabx} % for astronomical symbols of planets

% checkmark and crossmark
% requires pifont
\newcommand{\cmark}{\ding{51}}%
\newcommand{\xmark}{\ding{55}}%

% To draw arrows between given points
\newcommand\tikzmark[1]{%
  \tikz[remember picture,overlay]\node (#1) {};%
}

\newcommand\Connect[3][]{%
  \tikz[remember picture,overlay]
  \draw[-to,line width=1pt,>=latex,#1] (#2.north east) -- ( $ (#3.north west) + (-20pt,0) $ );%
}

\newcommand\Connectw[3][]{%
  \tikz[remember picture,overlay]
  \draw[-to,line width=3pt,>=latex,white!80!Yellow,#1] (#2.east) -- ( $ (#3.west) + (-2pt,0) $ );%
}

\newcommand\Connectr[3][]{%
  \tikz[remember picture,overlay]
  \draw[-to,line width=3pt,>=latex,red!80!Blue,#1] (#2.east) -- ( $ (#3.east) + (-2pt,0) $ );%
}

\newcommand\Connectlr[3][]{%
  \tikz[remember picture,overlay]
  \draw[-to,line width=3pt,>=latex,red,#1] (#2.east) -- ( $ (#3.east) + (-2pt,0) $ );%
}

\newcommand\Connectb[3][]{%
  \tikz[remember picture,overlay]
  \draw[-to,line width=5pt,>=latex,Blue,#1] (#2.east) -- ( $ (#3.east) + (-2pt,0) $ );%
}

\newcommand\ConnectCol[4][]{%
  \tikz[remember picture,overlay]
  \draw[-to,line width=4pt,>=latex,#1,#2] (#3.east) -- ( $ (#4.east) + (-2pt,0) $ );%
}


\newcommand{\hcancel}[1]{%
  \tikz[baseline=(tocancel.base)]{
    \node[inner sep=0pt,outer sep=0pt] (tocancel) {#1};
    \draw[red] (tocancel.south west) -- (tocancel.north east);
  }%
}%

\newcommand{\vdhdiagram}[9]{%Credits: Ruben Boots 2016
  \tikzset{
    cheating dash/.code args={on #1 off #2}{
      % Use csname so catcode of @ doesn't have do be changed.
      \csname tikz@addoption\endcsname{%
        \pgfgetpath\currentpath%
        \pgfprocessround{\currentpath}{\currentpath}%
        \csname pgf@decorate@parsesoftpath\endcsname{\currentpath}{\currentpath}%
        \pgfmathparse{\csname pgf@decorate@totalpathlength\endcsname}\let\rest=\pgfmathresult%
        \pgfmathparse{#1+#2}\let\onoff=\pgfmathresult%
        \pgfmathparse{max(floor(\rest/\onoff), 1)}\let\nfullonoff=\pgfmathresult%
        \pgfmathparse{max((\rest-\onoff*\nfullonoff)/\nfullonoff+#2, #2)}\let\offexpand=\pgfmathresult%
        \pgfsetdash{{#1}{\offexpand}}{0pt}}%
    }
  }
  \newboolean{RLOFone}
  \newboolean{RLOFtwo}
  \newboolean{CE}

  \def\mid{9}             % Horizontal position middle of system column

  \foreach
  \aone /\atwo /\xrone /\xrtwo /\xRLOFone /\xRLOFtwo /\xCE
  in {#1 / #2  / #3  / #4   / #5   / #6  / #7}
  {
    \pgfmathsetmacro{\yprev}{0} %dummy initialization
    \pgfmathsetmacro{\a}{max(\aone,\atwo)}  % Gets biggest RL-size for correct vertical placement
    \def\rone{\xrone*\aone}                 % Radius first star in terms of its RL size
    \def\rtwo{\xrtwo*\atwo}                 % Radius second star in terms of its RL size
    \def\y{\yprev-\a-.3 }                   % Vertical position system.

    \ifthenelse{\xRLOFone = 1}
    {\setboolean{RLOFone}{true}}
    {\setboolean{RLOFone}{false}}
    \ifthenelse{\xRLOFtwo = 1}
    {\setboolean{RLOFtwo}{true}}
    {\setboolean{RLOFtwo}{false}}
    \ifthenelse{\xCE = 1}
    {\setboolean{CE}{true}}
    {\setboolean{CE}{false}}

    \draw[dashed, line width=1pt]

    (\mid-\aone-0.2,\y)++(45:\aone) arc (45:315:\aone)
    (\mid+\atwo+0.2,\y)++(225:\atwo) arc (-135:135:\atwo)
    (\mid-\aone-0.2,\y)++(45:\aone) to[out=-45,in=135] ($ ({\mid+\atwo+0.2-\atwo/sqrt(2)},{\y-\atwo/sqrt(2)}) $)
    (\mid-\aone-0.2,\y)++(-45:\aone) to[out=45,in=-135] ($ ({\mid+\atwo+0.2-\atwo/sqrt(2)},{\y+\atwo/sqrt(2)}) $);


    \ifthenelse{\boolean{RLOFone}}
    {
      \draw[fill, color=#8]
      (\mid-\aone-0.2,\y) circle (\aone);
      \draw[fill, color=#9]
      (\mid+\atwo+0.2,\y) circle (\rtwo);
      \draw[fill, color=#8, opacity=0.5, draw opacity=0]
      (\mid-\aone-0.2,\y)++(45:\aone) to[out=-45,in=180] ($ ({\mid+\atwo+0.2},\y-\rtwo) $)
      to[out=180,in=45] ($ ({\mid-\aone-0.2+\aone/sqrt(2)},{\y-\aone/sqrt(2)}) $);
    }
    {
      \ifthenelse{\boolean{RLOFtwo}}
      {
        \draw[fill, color=#8]
        (\mid-\aone-0.2,\y) circle (\rone);
        \draw[fill, color=#9]
        (\mid+\atwo+0.2,\y) circle (\atwo);
        \draw[fill, color=#9, opacity=0.5, draw opacity=0]
        (\mid+\atwo+0.2,\y)++(135:\atwo) to[out=225,in=0] ($ ({\mid-\aone-0.2},\y-\rone) $)
        to[out=0,in=135] ($ ({\mid+\atwo+0.2-\atwo/sqrt(2)},{\y-\atwo/sqrt(2)}) $);
      }
      {
        \ifthenelse{\boolean{CE}}
        {
          \ifthenelse{\lengthtest{\aone pt > \atwo pt}}
          {
            \draw[fill, color=#8, opacity=0.5, draw opacity =0]
            (\mid-\aone-0.2,\y) circle (\aone+0.1)
            (\mid+\atwo+0.2,\y) circle (\atwo+0.1)
            (\mid+\atwo+0.2,\y+\atwo+0.1) to[out=180,in=-30] ($ ({\mid-\aone-0.2+(\aone+0.1)/2},{\y+(\aone+0.1)*sqrt(3)/2}) $)--
            ($ ({\mid-\aone-0.2+(\aone+0.1)/2},{\y-(\aone+0.1)*sqrt(3)/2}) $) to[out=30,in=180] ($ ({\mid+\atwo+0.2},{\y-\atwo-0.1}) $);
            \draw[fill, color=#8, overlay]
            (\mid-\aone-0.2,\y) circle (\rone)
            (\mid+\atwo+0.2,\y) circle (\rtwo);
          }
          {
            \ifthenelse{\lengthtest{\aone pt < \atwo pt}}
            {
              \draw[fill, color=#8, opacity=0.5, draw opacity =0]
              (\mid-\aone-0.2,\y) circle (\aone+0.1)
              (\mid+\atwo+0.2,\y) circle (\atwo+0.1)
              (\mid+\atwo+0.2,\y)++(120:\atwo+0.1) to[out=210,in=0] ($ ({\mid-\aone-0.2},{\y+(\aone+0.1)}) $)--
              ($ ({\mid-\aone-0.2},{\y-(\aone+0.1)}) $) to[out=0,in=150] ($ ({\mid+\atwo+0.2-(\atwo+0.1)/2},{\y-(\atwo+0.1)*sqrt(3)/2}) $);
              \draw[fill, color=#8]
              (\mid-\aone-0.2,\y) circle (\rone)
              (\mid+\atwo+0.2,\y) circle (\rtwo);
            }
            {
              \draw[fill, color=#8, opacity=0.5, draw opacity =0]
              (\mid-\aone-0.2,\y) circle (\aone+0.1)
              (\mid+\atwo+0.2,\y) circle (\atwo+0.1)
              (\mid+\atwo+0.2,\y)++(120:\atwo+0.1) to[out=210,in=-30] ($ ({\mid-\aone-0.2+(\aone+0.1)/2},{\y+(\aone+0.1)*sqrt(3)/2}) $)--
              ($ ({\mid-\aone-0.2+(\aone+0.1)/2},{\y-(\aone+0.1)*sqrt(3)/2}) $) to[out=30,in=150] ($ ({\mid+\atwo+0.2-(\atwo+0.1)/2},{\y-(\atwo+0.1)*sqrt(3)/2}) $);
              \draw[fill, color=#8]
              (\mid-\aone-0.2,\y) circle (\rone)
              (\mid+\atwo+0.2,\y) circle (\rtwo);
            }
          }
        }
      }
      {
        \draw[fill, color=#8]
        (\mid-\aone-0.2,\y) circle (\rone);
        \draw[fill, color=#9]
        (\mid+\atwo+0.2,\y) circle (\rtwo);
      }
    }


    \pgfmathparse{\y-\a}
    \global\let\yprev\pgfmathresult
  }
}

\newcommand{\vdhdiagramSN}[9]{%Credits: Ruben Boots 2016
  \tikzset{
    cheating dash/.code args={on #1 off #2}{
      % Use csname so catcode of @ doesn't have do be changed.
      \csname tikz@addoption\endcsname{%
        \pgfgetpath\currentpath%
        \pgfprocessround{\currentpath}{\currentpath}%
        \csname pgf@decorate@parsesoftpath\endcsname{\currentpath}{\currentpath}%
        \pgfmathparse{\csname pgf@decorate@totalpathlength\endcsname}\let\rest=\pgfmathresult%
        \pgfmathparse{#1+#2}\let\onoff=\pgfmathresult%
        \pgfmathparse{max(floor(\rest/\onoff), 1)}\let\nfullonoff=\pgfmathresult%
        \pgfmathparse{max((\rest-\onoff*\nfullonoff)/\nfullonoff+#2, #2)}\let\offexpand=\pgfmathresult%
        \pgfsetdash{{#1}{\offexpand}}{0pt}}%
    }
  }
  \newboolean{RLOFone}
  \newboolean{RLOFtwo}
  \newboolean{CE}

  \def\mid{9}             % Horizontal position middle of system column

  \foreach
  \aone /\atwo /\xrone /\xrtwo /\xRLOFone /\xRLOFtwo /\xCE
  in {#1 / #2  / #3  / #4   / #5   / #6  / #7}
  {
    \pgfmathsetmacro{\yprev}{0} %dummy initialization
    \pgfmathsetmacro{\a}{max(\aone,\atwo)}  % Gets biggest RL-size for correct vertical placement
    \def\rone{\xrone*\aone}                 % Radius first star in terms of its RL size
    \def\rtwo{\xrtwo*\atwo}                 % Radius second star in terms of its RL size
    \def\y{\yprev-\a-.3 }                   % Vertical position system.

    \ifthenelse{\xRLOFone = 1}
    {\setboolean{RLOFone}{true}}
    {\setboolean{RLOFone}{false}}
    \ifthenelse{\xRLOFtwo = 1}
    {\setboolean{RLOFtwo}{true}}
    {\setboolean{RLOFtwo}{false}}
    \ifthenelse{\xCE = 1}
    {\setboolean{CE}{true}}
    {\setboolean{CE}{false}}

    \ifthenelse{\boolean{RLOFone}}
    {
      \draw[fill, color=#8]
      (\mid-\aone-0.2,\y) circle (\aone);
      \draw[fill, color=#9]
      (\mid+\atwo+0.2,\y) circle (\rtwo);
      \draw[fill, color=#8, opacity=0.5, draw opacity=0]
      (\mid-\aone-0.2,\y)++(45:\aone) to[out=-45,in=180] ($ ({\mid+\atwo+0.2},\y-\rtwo) $)
      to[out=180,in=45] ($ ({\mid-\aone-0.2+\aone/sqrt(2)},{\y-\aone/sqrt(2)}) $);
    }
    {
      \ifthenelse{\boolean{RLOFtwo}}
      {
        \draw[fill, color=#8]
        (\mid-\aone-0.2,\y) circle (\rone);
        \draw[fill, color=#9]
        (\mid+\atwo+0.2,\y) circle (\atwo);
        \draw[fill, color=#9, opacity=0.5, draw opacity=0]
        (\mid+\atwo+0.2,\y)++(135:\atwo) to[out=225,in=0] ($ ({\mid-\aone-0.2},\y-\rone) $)
        to[out=0,in=135] ($ ({\mid+\atwo+0.2-\atwo/sqrt(2)},{\y-\atwo/sqrt(2)}) $);
      }
      {
        \ifthenelse{\boolean{CE}}
        {
          \ifthenelse{\lengthtest{\aone pt > \atwo pt}}
          {
            \draw[fill, color=#8, opacity=0.5, draw opacity =0]
            (\mid-\aone-0.2,\y) circle (\aone+0.1)
            (\mid+\atwo+0.2,\y) circle (\atwo+0.1)
            (\mid+\atwo+0.2,\y+\atwo+0.1) to[out=180,in=-30] ($ ({\mid-\aone-0.2+(\aone+0.1)/2},{\y+(\aone+0.1)*sqrt(3)/2}) $)--
            ($ ({\mid-\aone-0.2+(\aone+0.1)/2},{\y-(\aone+0.1)*sqrt(3)/2}) $) to[out=30,in=180] ($ ({\mid+\atwo+0.2},{\y-\atwo-0.1}) $);
            \draw[fill, color=#8, overlay]
            (\mid-\aone-0.2,\y) circle (\rone)
            (\mid+\atwo+0.2,\y) circle (\rtwo);
          }
          {
            \ifthenelse{\lengthtest{\aone pt < \atwo pt}}
            {
              \draw[fill, color=#8, opacity=0.5, draw opacity =0]
              (\mid-\aone-0.2,\y) circle (\aone+0.1)
              (\mid+\atwo+0.2,\y) circle (\atwo+0.1)
              (\mid+\atwo+0.2,\y)++(120:\atwo+0.1) to[out=210,in=0] ($ ({\mid-\aone-0.2},{\y+(\aone+0.1)}) $)--
              ($ ({\mid-\aone-0.2},{\y-(\aone+0.1)}) $) to[out=0,in=150] ($ ({\mid+\atwo+0.2-(\atwo+0.1)/2},{\y-(\atwo+0.1)*sqrt(3)/2}) $);
              \draw[fill, color=#8]
              (\mid-\aone-0.2,\y) circle (\rone)
              (\mid+\atwo+0.2,\y) circle (\rtwo);
            }
            {
              \draw[fill, color=#8, opacity=0.5, draw opacity =0]
              (\mid-\aone-0.2,\y) circle (\aone+0.1)
              (\mid+\atwo+0.2,\y) circle (\atwo+0.1)
              (\mid+\atwo+0.2,\y)++(120:\atwo+0.1) to[out=210,in=-30] ($ ({\mid-\aone-0.2+(\aone+0.1)/2},{\y+(\aone+0.1)*sqrt(3)/2}) $)--
              ($ ({\mid-\aone-0.2+(\aone+0.1)/2},{\y-(\aone+0.1)*sqrt(3)/2}) $) to[out=30,in=150] ($ ({\mid+\atwo+0.2-(\atwo+0.1)/2},{\y-(\atwo+0.1)*sqrt(3)/2}) $);
              \draw[fill, color=#8]
              (\mid-\aone-0.2,\y) circle (\rone)
              (\mid+\atwo+0.2,\y) circle (\rtwo);
            }
          }
        }
      }
      {

        \draw[fill, color=#8]
        (\mid-\aone-0.2,\y) circle (\rone);
        \draw[color=#8,
        pattern=crosshatch, pattern
        color=red]
        (\mid-\aone-0.2,\y) circle (\rone);
        \draw[fill, color=#9]
        (\mid+\atwo+0.2,\y) circle (\rtwo);
      }
    }

    \draw[dashed, line width=1pt]

    (\mid-\aone-0.2,\y)++(45:\aone) arc (45:315:\aone)
    (\mid+\atwo+0.2,\y)++(225:\atwo) arc (-135:135:\atwo)
    (\mid-\aone-0.2,\y)++(45:\aone) to[out=-45,in=135] ($ ({\mid+\atwo+0.2-\atwo/sqrt(2)},{\y-\atwo/sqrt(2)}) $)
    (\mid-\aone-0.2,\y)++(-45:\aone) to[out=45,in=-135] ($ ({\mid+\atwo+0.2-\atwo/sqrt(2)},{\y+\atwo/sqrt(2)}) $);

    \pgfmathparse{\y-\a}
    \global\let\yprev\pgfmathresult
  }
}

\newcommand{\udef}{\stackrel{\mathrm{def}}{=}}

\newcommand{\todo}[1]{{\Huge $\blacksquare$~\textbf{\color{red}#1}~$\blacksquare$}\\ }

\newenvironment<>{varblock}[2][.9\textwidth]{%
  \setlength{\textwidth}{#1}
  \begin{actionenv}#3%http://www.latex-community.org/forum/viewtopic.php?f=4&t=4203
    \def\insertblocktitle{#2}%
    \par%
    \usebeamertemplate{block begin}}
  {\par%
    \usebeamertemplate{block end}%
  \end{actionenv}}

% some colors
\definecolor{bblue}{HTML}{144b7a}%unipi logo color
\definecolor{blue_mesa}{rgb}{58,86,253}%mesa logo color
\definecolor{gray}{gray}{0.}
\definecolor{Yellow}{HTML}{FFFF99}
\definecolor{Green}{rgb}{0,0.6,0}
\definecolor{Orange}{HTML}{FF9900}
\definecolor{Mauve}{rgb}{0.58,0,0.82}
\definecolor{lightred}{HTML}{FF3333}
\definecolor{redinplot}{HTML}{FF0000}
\definecolor{yellowinplot}{HTML}{BFBF00}
\definecolor{cyaninplot}{HTML}{00BFBF}
\definecolor{reddish}{HTML}{FF8484}
\definecolor{yellowish}{HTML}{FFF684}
\definecolor{pinkish}{HTML}{FF94FF}
\colorlet{whiteish}{white!80!Yellow}
\colorlet{lightgray}{gray!50}

%% set theme
\usetheme{metropolis}   %%% Theme Name
% customizations
\metroset{progressbar=frametitle}
\metroset{subsectionpage=progressbar}
\setbeamertemplate{blocks}[rounded]
% colors
\usecolortheme{seahorse}
\setbeamercolor{structure}{fg=Blue, bg=white}
\setbeamercolor{progress bar}{fg=Blue, bg=gray!30}
\setbeamercolor{progress bar in section page}{fg=Blue, bg=gray!30}
\setbeamercolor{frametitle}{bg=, fg=Blue} %% set colors

%block
\setbeamercolor{block body}{use=structure,bg=white!80!Yellow}
\setbeamercolor{block title}{use=Blue,fg=Blue,bg=white!80!Yellow}

% \setbeamercolor{block body}{bg=white,fg=black} %% white background
% \setbeamercolor{block title}{bg=white,fg=Blue} %% white background
\setbeamercolor{background canvas}{bg=,fg=black} %% white background
% remove margins
\setbeamersize{text margin left=2pt,text margin right=2pt}
\setbeamertemplate{section in toc}{
  \centering \bf \color{Blue}\inserttocsection \par}
\setbeamertemplate{subsection in toc}{
  \centering \inserttocsubsection \par}

%%% Local Variables:
%%% mode: plain-tex
%%% TeX-master: t
%%% End:


% uncomment these for adding comments for secondary screen with
% \note
% \setbeameroption{show notes}
\setbeameroption{show notes on second screen=right}
%%%%%%%%%%%%%%%%%%%%%%%%%%%%%%%%%%%%%%%%%%%%%%%%%%%%%%%%%%%%%%%%%%%%%%%%%%%%%%%%%%%%%%
%%%% \setbeamertemplate{background}[grid][step=10]        %%%%Grid on slides to help positioning
%%%%%%%%%%%%%%%%%%%%%%%%%%%%%%%%%%%%%%%%%%%%%%%%%%%%%%%%%%%%%%%%%%%%%%%%%%%%%%%%%%%%
%%%% Global Background must be put in preamble
%%%% \setbeamercolor{background canvas}{bg=BGcol}

\begin{document}


% Massive stars, those with more than ten times the mass of our Sun,
% live short and tumultuous lives ending in dramatic supernova
% explosions that leave behind neutron stars or black holes. These
% stars are typically not evolving alone: most have one (or more)
% companion stars orbiting around them at distances such that as the
% stars evolve, they will "touch each other," exchange mass, and
% sometimes even merge. These interactions can modify the appearance,
% evolution, and fate of both stars in the binary, with implications
% for the galaxy they reside in and the compact objects they produce.
% In most cases, when the first star reaches the end of its life and
% explodes as a supernova, it ejects the surviving "widowed" companion
% at speeds of several tens of km/s–these massive stars live fast in a
% very literal sense. Join us in this seminar as Dr. Mathieu Renzo
% guides teachers through a scientific exploration of the life and
% death of massive binary stars over the last 2000 years, reviewing
% the basics of stellar evolution, making connections to the visible
% night sky, and sharing observations across the electromagnetic
% spectrum and in gravitational waves. In the second half of the
% seminar, teachers will continue with small group discussions as they
% dive into the content and data presented. This seminar is best
% suited for physics and mathematics teachers, but all are welcome.

% Dr. Mathieu Renzo is a stellar physicist at the Center for
% Computational Astrophysics of the Simons Foundation Flatiron
% Institute. His research focuses on the double-sided relation between
% the evolution of massive stars in binary systems and their final
% explosions.




\graphicspath{{./fig_slides/}}

% %%%% Tikz stuff
% %%%% For every picture that defines or uses external nodes, you'll have to
% %%%% apply the 'remember picture' style. To avoid some typing, we'll apply
% %%%% the style to all pictures.
% \tikzstyle{every picture}+=[remember picture, overlay]
% %%%% By default all math in TikZ nodes are set in inline mode. Change this to
% %%%% displaystyle so that we don't get small fractions.
% %%%% \everymath{\displaystyle}


\bgroup
\metroset{progressbar=none}
\setbeamertemplate{background}{\includegraphics[width=\paperwidth,height=\paperheight,keepaspectratio]{zoph_onethird}}
\begin{frame}[plain]
  \begin{textblock}{0.5}[0.5,0.5](0.66,0.25)
    % \begin{tcolorbox}[boxrule=0pt,standard jigsaw, colback=black,
    %   left=0pt, right=0pt, top=0pt, bottom=0pt,
    %   colframe=black,
    %   opacityback=0.2]
    \centering
    \textbf{\textcolor{white!80!Yellow}{{\Large Live fast and die
          young:\\}
        Massive, exploding, and speeding stars}}
    % \end{tcolorbox}
  \end{textblock}

  \begin{textblock}{1}[0,1](0.01,.99)% \centering
    \textcolor{white!80!Yellow}{\tiny
      Collaborators:~Y.~G\"otberg, E.~Zapartas, S.~Justham,
      K.~Breivik, L.~van~Son, R.~Farmer, M.~Cantiello,
      B.~D.~Metzger, D.~Hendricks,
      C.~Xin, E.~Farag, S.~Oey, S.~de~Mink, ...}
  \end{textblock}

  \begin{textblock}{0.5}[0.5, 0.5](0.66, 0.85)
    \centering
    \textcolor{white!80!Yellow}{\textbf{\large Mathieu
        Renzo}}\linebreak  \textcolor{whiteish}{\footnotesize\href{mailto:mrenzo@flatironinstitute.org}{mrenzo@flatironinstitute.org}}
  \end{textblock}

  \begin{textblock}{0.2}[1, 0](0.99, 0.01)
    \includegraphics[width=\textwidth]{CCA_logo_color}
  \end{textblock}
\end{frame}
\egroup

\bgroup
\metroset{progressbar=none}
\setbeamertemplate{background}{\includegraphics[width=\paperwidth, keepaspectratio]{where_i_am_from}}
\begin{frame}[plain]

\end{frame}
\egroup

\begin{frame}
  \frametitle{Rules of engagement}
  \todo{text from Uzma}
\end{frame}

\begin{frame}
  %%% So how do we study this process of stellar explosions? This is
  %%% the perfect example of how theoretical astrophysics work: you
  %%% make a physical model, write down the equations, plug them into
  %%% a super-computer and obtain a numerical simulation. This however
  %%% requires many layers of abstraction: how we write the equations
  %%% and how we tell the computer to solve the equations matters a
  %%% lot for the kind of answer that you get.

  %%% Nature doesn't need to do all of this, it does not need to know
  %%% how stars explode to make them explode! Therefore, and this is
  %%% not true just to understand how stars explode, but in general
  %%% for any theoretical astrophysics problem, we need to compare to
  %%% observations to avoid fooling ourselves!
  \frametitle{I am a theoretical and computational astrophysicist}
  \centering
  \begin{textblock}{0.35}[0.5,0](0.18,0.16)
    \Large  \textcolor{red}{Theory} +
  \end{textblock}

  \includegraphics[height=0.7\textheight]{Cartesius_0}\\
  \begin{textblock}{0.4}[0.5,1](0.64,0.835)
    \color{black} \tiny Dutch national supercomputer Cartesius
  \end{textblock}

  \begin{textblock}{0.35}[0.5,0.0](0.8,0.85)
    \Large =
  \end{textblock}

  \begin{textblock}{0.5}[0.5,1](0.8,0.95)
    \textcolor{red}{ \Large Numerical Simulations\phantom{A}}
  \end{textblock}


  \begin{textblock}{0.4}[0.5,0.5](0.5,0.5)
    \includegraphics[width=0.75\textwidth]{binary}
  \end{textblock}

\end{frame}

\begin{frame}
  \note{Teachers have a long lasting impact on the communities they
    work with}
  \frametitle{My main goals today}


  \begin{textblock}{0.9}[0.5,0.5](0.5,0.5)
    \centering
    \begin{enumerate}\Large
    \item Share my research\\[10pt]
    \item \emph{\underline{Experiment}} together with high-school teachers\\[10pt]
      \begin{itemize}\Large
      \item Connect research, outreach, and education
      \item Leverage \textbf{open science}
      \end{itemize}
    \item<2>[\textbf{Q.}] \emph{What would \underline{you} want from
        researchers?}
    \end{enumerate}

  \end{textblock}

\end{frame}


\section{\textcolor{Blue}{Stellar evolution quickstart}}


\bgroup
\setbeamercolor{background canvas}{bg=black}
\begin{frame}{\color{whiteish} The nearest star to us is the Sun}
\note{The main thing that is special to astrophysicists about the Sun
  is that it is much much closer than any other star.}
  \begin{textblock}{0.45}[0,0.5](0,0.55)
    \includegraphics[width=\linewidth]{./fig_slides/physics1.jpg}
  \end{textblock}

  \begin{textblock}{0.65}[0.5,0.5](0.75,0.5)
    \begin{itemize}
      \Large
    \item[\color{red}\textbullet] \textcolor{whiteish}{Current age $\sim 4.5\cdot 10^9$\,yr}
    \item[\color{red}\textbullet] \textcolor{whiteish}{Expected lifetime $\sim 10^{10}$\,yr}
    \item[\color{red}\textbullet]
      \textcolor{whiteish}{$L_\odot\simeq 4\cdot 10^{33}\,\mathrm{erg\ s^{-1}}$}
    \item[\color{red}\textbullet]
     \textcolor{whiteish}{$M_\odot\simeq 2\cdot 10^{33}\,\mathrm{g}\simeq 10^4\,M_\Jupiter \sim3\cdot 10^5\,M_\Earth$}

    \item[\color{red}\textbullet] \textcolor{whiteish}{$R_\odot \simeq 6.95\cdot 10^{10}$\,cm}
    \end{itemize}
  \end{textblock}


  \begin{textblock}{1}[0,1](0.01,0.99)
    \textcolor{gray!50}{\tiny credits: NASA}
  \end{textblock}
\end{frame}



\begin{frame}{\textcolor{whiteish}{Stars are large balls of matter that ``resist'' their own weight}}
      \note{Analogy: if someone pushes a door, and you are behind and
        try to keep the door closed, it takes you energy.\\

        Similarly the weight of a star tries to crush it to a point, but
        the star resist. But to do so the star has to spend energy}

    \begin{textblock}{0.45}[0,0.5](0,0.55)
      \includegraphics[width=\linewidth]{./fig_slides/physics_gravity.jpg}
    \end{textblock}

  \begin{textblock}{0.5}[0.5,0](0.75,0.16)
    \centering
    \textcolor{whiteish}{
      Pushing against gravity \\
      costs energy}
  \end{textblock}

\end{frame}


\begin{frame}{\textcolor{whiteish}{Stars produce their own energy by nuclear fusion}}
      \note{
        \newline \textcolor{red}{Nucleosynthesis} deep
        \textbf{inside}, in the \textbf{center} which we usually call
        ``core''.

        Fe cannot release energy through nuclear fusion

      }

      \begin{textblock}{0.5}[0.5,0](0.75,0.16)
        \centering
        \textcolor{whiteish}{Pushing against gravity \\
          costs energy}\\[10pt]
        % \begin{center}
        %   \begin{sideways}
        %     \Leftrightarrow
        %   \end{sideways}
        % \end{center}
        \textcolor{whiteish}{Energy leaks as $\gamma$ (and $\nu$)\\[10pt]
          To compensate,\\
          stars produce energy by\\
          \textbf{nuclear fusion}\\[10pt]
          \includegraphics[width=0.3\textwidth]{Nucleosynthesis_fuel}\\
          When they run out of ``\textcolor{Orange}{fuel}'' \\
          and are forced to evolve}
      \end{textblock}

    \begin{textblock}{0.45}[0,0.5](0,0.55)
      \includegraphics[width=\linewidth]{./fig_slides/physics_star.jpg}
    \end{textblock}
\end{frame}
\egroup


\begin{frame}[c]
  \frametitle{Core burns heavier fuel, shells burn the leftovers}
  \begin{textblock}{0.75}[0.,0.0](0.01,0.12)
    \includegraphics[width=0.54\textwidth]{binding_nuc}
  \end{textblock}

  \begin{textblock}{1}[1,1](1.30,0.99)
    \includegraphics[width=0.7\textwidth]{onion.png}
  \end{textblock}

  \begin{textblock}{0.4}[0.5,0.5](0.755,0.56)
    \begin{rotate}{30}
      \centering
      ``Onion'' structure
    \end{rotate}
  \end{textblock}

  \begin{textblock}{1}[0,1](0.01,0.99)
    \textcolor{gray!50}{\tiny Renzo 2015}\hfill\,
  \end{textblock}
\end{frame}

\begin{frame}
  \frametitle{Stars are like people: they get bigger as they grow}
  \only<1>{\includegraphics[height=\textheight]{16MsunRadius_empty}
  \note{Introduce log-scale}}
  \only<2>{\includegraphics[height=\textheight]{16MsunRadius_MS}}
  \only<3-4>{\includegraphics[height=\textheight]{16MsunRadius_no_annotations}}

  \only<2>{
    \begin{textblock}{0.4}[0.5,0.5](0.75,0.5)
      \centering
      \textbf{Main sequence:}\\
      H$\rightarrow$ He, 90\% of stellar lifetime
    \end{textblock}
  }

  \only<2-4>{
    \begin{textblock}{0.1}[0.5,0.5](0.15,0.75)
      \includegraphics[width=\textwidth]{MS}
    \end{textblock}
  }

  \only<3>{
    \begin{textblock}{0.4}[0.5,0.5](0.75,0.5)
      \centering
      \textbf{Red (super) giant:}\\
      Core+shell burning, 10\% of stellar lifetime
    \end{textblock}
  }

  \only<3-4>{
    \begin{textblock}{0.3}[0.5,0.5](0.28,0.35)
      \includegraphics[width=\textwidth]{RSG}
    \end{textblock}
  }


%   \only<4>{
%     \begin{textblock}{0.55}[1,0.5](1.05,0.5)
%       \includegraphics[width=\textwidth]{onion.pdf}
%     \end{textblock}
%     \frametitle{Only the center is hot enough to burn heavy fuel}
%     \begin{textblock}{0.4}[0.,0.5](0.5,0.35)
%       \centering
%       ``Onion'' structure
%     \end{textblock}


%     \note{Fe cannot release energy through nuclear fusion}
% }


  \only<4>{
    \begin{textblock}{0.2}[0.5,0.5](0.53,0.27)
      \includegraphics[width=\textwidth]{SN}
    \end{textblock}

    \begin{textblock}{0.4}[0.5,0.5](0.75,0.53)
      \centering
      \textbf{Core collapse} and possibly \textbf{supernova
        explosion}:\\
      $\tau(M)\propto M^{-\alpha} \simeq 3-50$\,Myr\\
      \hfill\textcolor{gray!50}{\tiny e.g., Zapartas et al. 2017}
    \end{textblock}

  }



  \begin{textblock}{0.3}[0.5,1](0.75,0.99)
    \centering
    \small
    \color{gray!50}{\tiny Simulation output from}\\
    \includegraphics[width=0.35\textwidth]{MESA}
  \end{textblock}
\end{frame}





\AtBeginSection[]{}
\section*{\textcolor{Blue}{Massive stars drive the evolution of the Universe}}


\bgroup
\metroset{progressbar=none}
\setbeamertemplate{background}{\includegraphics[width=\paperwidth,height=\paperheight,keepaspectratio]{zoph_onethird}}
\begin{frame}
  \color{whiteish}
  \only<1>{
    \note{\textbf{But why would you care about massive stars?}
      This is $\zeta$ Ophiuchi, the nearest 20$M_\odot$ star to
      Earth, I'll explain later why it looks this way.

      Massive stars like this one are really central to most fields in
      astrophysics, and despite that they still pose significant
      challenges.\\

      Be upfront: Massive $\gtrsim 7.5\,M_\odot$ = early B and O type stars
    }
    \frametitle{\textcolor{whiteish}{Why massive stars?~{\footnotesize ($M_\mathrm{initial} \gtrsim 7.5\,M_\odot$)}}}
    \begin{textblock}{0.66}[0.,0.5](0.35,0.55)
      \textcolor{whiteish}{\large \bf $\zeta$ Ophiuchi is the nearest massive star to Earth}
    \end{textblock}
  }



  \tikz[remember picture, overlay]\node [xshift=85pt, yshift=-13pt] at (current page.west)  (O) {};
  \only<2>{
    \note{As I said, they produce the compact objects in their
      explosions, and these explosions are some of the most
      powerful phenomena available to observations, but even
      during their life, they stir and pollute their surrounding
      gas and have a huge influence on the evolution of galaxies
      (their composition, how they form stars and clusters) and
      the universe as a whole, such as during the epoch of
      ionization\\

      \todo{Mention time-domain, GW, and multimessenger and connection}
    }


    \frametitle{\textcolor{whiteish}{They are the progenitors of
        neutron stars \& black holes}}


    \begin{textblock}{0.4}[0.5,0.](0.75,0.18) \tiny
      \textcolor{gray!50}{
        \emph{EM}:\\[-1pt]
        O'Connor \& Ott 2011, Ertl \emph{et al.} 2016, 2020, Farmer
        \emph{et al.} 2016, Morozova \emph{et al.}
        2015, 2016, Renzo \emph{et al.} 2017, 2020a,
        b, c, Laplace \emph{et al.}  2021, Vartanyan
        \emph{et al.}  2021, Zapartas \emph{et al.}
        2017a, 2019, 2021a, b, Marchant \emph{et al.}
        2019, Farmer \emph{et al.}  2019,
        2020}, ...\\[5pt]
      \textcolor{gray!50}{
        \emph{GW}:\\[-1pt]
        LVK collaboration 2015-23, Vigna-G\'omez \emph{et al.}
        2018, van~Son \emph{et al.}
        2020, 2021, Callister, Renzo, Farr 2021, Renzo \emph{et al.}
        2021}, ...
    \end{textblock}
  }


  \only<2-3>{
    \tikz[remember picture, overlay]\node [xshift=-30pt,yshift=60pt, text width=
    90pt] at (current page.center)
    (SNe) {\centering
      \includegraphics[width=50pt]{BHNS.png}\\
      \centering
      \small
      Compact objects\\
      \& transients\\
      (incl.~GW)\\
    };
    \Connectw{O}{SNe}
  }

  \only<3>{
    \note{\todo{Mention Katlyn Kratter's work on SF}}
    \frametitle{\textcolor{whiteish}{They shape their environment \& the Universe as a whole}}

    \tikz[remember picture, overlay]\node [xshift=140pt,yshift=15pt, text width=
    180pt] at (current page.center)
    (SF) {
      \includegraphics[width=50pt]{reion}\\
      \centering
      \small
      Ionizing rad.
      % \\[-3pt]
      % \hfill \textcolor{gray!50}{\tiny Hopkins \emph{et al.} 2014}
    };


    \tikz[remember picture, overlay]\node [xshift=160pt,yshift=-55pt, text width=
    90pt] at (current page.center)
    (Ioniz) {\centering
      \includegraphics[width=50pt]{star_form_cluster}\linebreak
      \small Star Formation\\
      \& cluster evolution\\
      % \hfill \textcolor{gray!50}{\tiny Lada \& Lada 2003}
    };

    \tikz[remember picture, overlay]\node [xshift=40pt,yshift=-95pt, text width=
    90pt] at (current page.center)
    (Nucl) {\centering\includegraphics[width=50pt]{nucleosynthesis}\\
      \centering
      \small
      Nucleosynthesis \&\\
      chemical evolution\\
      % \hfill \textcolor{gray!50}{\tiny Larson 1974, Nomoto \emph{et al.} 2013}
    };


    \Connectw{O}{Nucl}
    \Connectw{O}{SF}
    \Connectw{O}{Ioniz}

    \begin{textblock}{0.01}[0.5,0.5](0.75,0.2)
      \begin{rotate}{-30}
        \centering
        \textcolor{whiteish}{\bf Stellar feedback}
      \end{rotate}
    \end{textblock}
  }


  \begin{textblock}{0.33}[0,1](0.01,0.99)
    \textcolor{gray!50}{\tiny Spitzer, NASA/JPL}
  \end{textblock}


\end{frame}
\egroup




\begin{frame}
  \note{So why binaries? well it turns out that wherever we look,
    young massive stars almost always have companions.\\

    This plot shows the fraction of stars that have at least one
    companion in bold, and two or more companions (so triple, quadruple,
    etc.) as a function of mass. For massive stars, 70-100\% have one
    companion, and the majority has more than one companion.\\

    Not shown here is the distribution of separations, but accounting
    for that, we expect that 70\% of young O-type hot stars will
    exchange mass with a companion before they die.
  }
  \frametitle{Most massive stars are born with companion(s)}
  \vspace*{-2pt}
  \includegraphics[height=\textheight]{bin_frac}

  \begin{textblock}{0.33}[0.5,0.5](0.8,0.55)
    \begin{block}{\centering Binary interactions\\ are \textcolor{red}{common}}
      \centering
      \includegraphics[width=\textwidth]{binary}\\[-1pt]
      \phantom{\tiny a}
    \end{block}
  \end{textblock}
  \begin{textblock}{0.32}[0.5,0.5](0.8, 0.74)
    \begin{center}
      \textcolor{whiteish}{70\% of massive stars}\\[-2pt]
      \hfill\textcolor{gray!50}{\tiny Sana \emph{et al.} 2012}\,
    \end{center}
  \end{textblock}

  \begin{textblock}{0.3}[0.5,0.5](0.265, 0.77)
    \begin{rotate}{28}
      \footnotesize \textbf{At least one companion}
    \end{rotate}
  \end{textblock}

  \begin{textblock}{0.3}[0.5,0.5](0.455, 0.85)
    \begin{rotate}{32}
      \footnotesize Two or more companions
    \end{rotate}
  \end{textblock}


  \begin{textblock}{0.5}[0, 1](0.01, 0.99)
    \textcolor{gray!50}{\tiny modified from Offner \emph{et al.} 2022}
  \end{textblock}
  \begin{textblock}{0.75}[1, 1](0.99, 0.99)
    \hfill\textcolor{gray!50}{\tiny
      see also Mason \emph{et al.} 2010, Kobulnicky \& Fryer 2007,
      Moe \& di Stefano 2017}\,
  \end{textblock}
\end{frame}

\bgroup
\metroset{progressbar=none}
\setbeamercolor{background canvas}{bg=black}
\begin{frame}
\note{  But let's start with something a bit more familiar. This is a
  constellation of the northern sky which probably most of you
  know. You can use it to find the direction of north by
  taking this side and prolonging fine times in it's direction to
  find the Polaris.

  This constellation was used in ancient Greece as an entry-test
  to become a student of astronomy. The question is how many
  stars are in this constellation?

  In reality there are more: Mizar and Alcor are a visual binary,
  that is two stars orbiting each other in a binary system, and it
  is such that if you have a very good sight, you can distinguish
  them by naked eye. The greeks used this to test the sight of
  perspective students.

  In reality things are even more complicated, infact Mizar itself
  is a quadruple system (but these stars you cannot distinguish
  with your naked eye). Alcor itself in 2009 has been claimed as a
  binary, so this system in the end is actually three pairs of
  binaries!
  Stars are typically born in binary or multiple systems.

  Broadly speaking, what I study is how this fact changes the
  evolution of stars, and how they die.
}
\includegraphics[height=\paperheight]{fig_slides/zoom0}

  \begin{textblock}{0.45}[0.5,0.5](0.75,0.75)
    \centering
    \textcolor{whiteish}{\huge The big dipper}
  \end{textblock}
\end{frame}

\begin{frame}
  \includegraphics[height=\paperheight]{fig_slides/zoom1}

    \begin{textblock}{0.45}[0.5,0.5](0.75,0.75)
    \centering
    \textcolor{whiteish}{\huge Mizar \& Alcor}
  \end{textblock}

\end{frame}

\begin{frame}
  \includegraphics[height=\paperheight]{fig_slides/zoom2}
  \begin{textblock}{0.45}[0.5,0.5](0.75,0.75)
    \begin{tcolorbox}[boxrule=0pt,standard jigsaw, colback=black,
      left=0pt, right=0pt, top=0pt, bottom=0pt,
      colframe=black,
      opacityback=0.2]
      \color{whiteish}
      \centering \bf \Large Most stars are in binaries\\ or multiple systems
      \end{tcolorbox}
    \end{textblock}
\end{frame}


\begin{frame}[c]
  \frametitle{\textcolor{whiteish}{Binary interactions \emph{change} massive star feedback}}
  \note{And binary interactions can cause the stars to evolve
    differently, and change what compact objects they make, if and how
    they explode, where and when they explode, and all this changes the
    role they play in the universe}
  \vspace*{0.25\textheight}
  \includegraphics[width=0.3\textwidth]{bin_small}
  \begin{textblock}{0.3}[0,0](0.01,0.78)
    \textcolor{gray!50}{\tiny ESO, L. Calçada, M. Kornmesser, S.E. de
      Mink}
  \end{textblock}
  {
    \color{whiteish}
    \tikz[remember picture, overlay]\node [xshift=85pt, yshift=-13pt] at (current page.west)  (O) {};

    \tikz[remember picture, overlay]\node [xshift=-30pt,yshift=60pt, text width=
    90pt] at (current page.center)
    (SNe) {\centering
      \includegraphics[width=50pt]{BHNS.png}\\
      \centering
      \small
      Compact objects\\
      \& transients\\
      (incl.~GW)\\
    };
    \Connectw{O}{SNe}
    \tikz[remember picture, overlay]\node [xshift=140pt,yshift=15pt, text width=
    180pt] at (current page.center)
    (SF) {
      \includegraphics[width=50pt]{reion}\\
      \centering
      \small
      Ionizing rad.
      % \\[-3pt]
      % \hfill \textcolor{gray!50}{\tiny Hopkins \emph{et al.} 2014}
    };


    \tikz[remember picture, overlay]\node [xshift=160pt,yshift=-55pt, text width=
    90pt] at (current page.center)
    (Ioniz) {\centering
      \includegraphics[width=50pt]{star_form_cluster}\linebreak
      \small Star Formation\\
      \& cluster evolution\\
      % \hfill \textcolor{gray!50}{\tiny Lada \& Lada 2003}
    };

    \tikz[remember picture, overlay]\node [xshift=40pt,yshift=-95pt, text width=
    90pt] at (current page.center)
    (Nucl) {\centering\includegraphics[width=50pt]{nucleosynthesis}\\
      \centering
      \small
      Nucleosynthesis \&\\
      chemical evolution\\
      % \hfill \textcolor{gray!50}{\tiny Larson 1974, Nomoto \emph{et al.} 2013}
    };


    \Connectw{O}{Nucl}
    \Connectw{O}{SF}
    \Connectw{O}{Ioniz}

    \begin{textblock}{0.01}[0.5,0.5](0.75,0.2)
      \begin{rotate}{-30}
        \centering
        \textcolor{whiteish}{\bf Stellar feedback}
      \end{rotate}
    \end{textblock}
  }

\end{frame}
\egroup


\begin{frame}
  \frametitle{Stars move through their host galaxy: peculiar motion on
  top of orbit}
  \includegraphics[height=0.9\textheight]{tetzlaff11_marked}
  \begin{textblock}{1}[0.5,0](0.4,0.92)
    \centering
    \color{black}$v_\mathrm{3D} \ \mathrm{[km\ s^{-1}]}$
  \end{textblock}

    \begin{textblock}{0.35}[1,0.5](0.98,0.35)
      \centering
      \setbeamercolor{block title}{use=structure,fg=Blue,bg=white!80!Yellow}
      \setbeamercolor{block body}{use=structure,fg=black,bg=white!80!Yellow}
      \begin{block}{\centering Runaways are the tail}
        Typically have fewer companions\\[5pt]
        {\tiny \hfill \textcolor{gray!50}{Blaauw 61}\,}
      \end{block}
    \end{textblock}
    \only<2>{
      \frametitle{\textcolor{red}{Runaway} stars are common among massive stars}
      \note{ what is typically done is to ARBITRARILY draw a line to give an
        effective definition of what ``tail'' is: everything on the right is
        defined as a runaway star. Typical threshold (no physical meaning) are
        30 or 40 $kms$.I use $30kms$. For O type stars about 10-30\% of stars
        are in the tail, and this is comparable to Be stars (and maybe slightly
        more than all early B-type stars), although these fractions are very
        uncertain because of the difficulty of measuring the 3 components of the
        velocity of a star and defining what is the tail of the distribution.\\

        Note that HYPERvelocity star, with velocities larger than the escape
        velocity from the Galaxy have also been found, but these require
        different ejection mechanisms I will not touch upon.
      }
    \begin{textblock}{0.32}[1,0.5](0.98,0.6)
      \centering
      \setbeamercolor{block title}{use=structure,fg=Blue,bg=white!80!Yellow}
      \setbeamercolor{block body}{use=structure,fg=black,bg=white!80!Yellow}
      \begin{block}{\centering Observed fraction}
        \centering
        $M\gtrsim 15\,M_\odot,\,\sim10-20\%$
        \vspace{0.5em}
      \end{block}
    \end{textblock}
  }

  \begin{textblock}{1}[0.,1](0.01,0.99)
    \textcolor{gray!50}{\tiny Tetzlaff \emph{et al.} 2011}
  \end{textblock}

\end{frame}


\begin{frame}
  \frametitle{How to measure stellar velocities \textcolor{red}{?}}

\end{frame}


\begin{frame}
  \frametitle{How to measure stellar velocities}
  \note{  but how do we know that a significant fraction of massive stars
    are runaways? To measure their velocities is not always easy. We
    can have bow shocks if they move supersonically compared to the
    ambient ISM, which can be used to constrain the velocity. We can
    use the Doppler shift of spectral lines to measure their velocity
    along the line of sight, and IF their distance is known, we can in
    principle also use their proper motion, to measure the transverse
    velocity of the star. I care to mention this because the future
    Gaia data release will provide distances (through parallax) and
    proper motions of many more stars, potentially increasing the
    sample of runaways known in the galaxy.
  }
  \only<1>{
    \begin{textblock}{0.35}[0.5,0.5](0.75,0.6)
      \centering
      \textcolor{black}{``Bow wave''}\\[-3pt]
      \textcolor{black}{$v_\mathrm{boat} > v_\mathrm{wave}$}\\
      \includegraphics[width=\textwidth]{bow_wave_cargo}
    \end{textblock}
    \begin{textblock}{0.25}[0.5,0](0.15,0.6)
      \centering
      \textcolor{black}{``Bow shock''}\\[-3pt]
      \textcolor{black}{$v>c_s$}
    \end{textblock}
  }

  \begin{center}
    \begin{minipage}[hb]{0.25\linewidth}
      \vspace*{-95pt}
      \includegraphics[width=\textwidth]{./fig_slides/Zoph}\\
    \end{minipage} \vspace*{30pt}
    \begin{minipage}[hb]{0.4\linewidth}
      \begin{itemize}\Large
      \item<1->[$\Leftarrow$] Special features\\
        {\small \centering (\textcolor{red}{if} relatively \textcolor{red}{nearby})}
        \vspace*{5pt}
      \item<2->[] Doppler shifts \hfill
        \textcolor{Blue}{$\Rightarrow$}
        \vspace*{5pt}
      \item<3>[] Proper motions\\
        {\small \centering (\textcolor{red}{if} distance known)}
        \begin{center}
          \begin{sideways}
            $\Leftarrow$
          \end{sideways}
          \linebreak \hspace*{-20pt}\includegraphics[width=0.7\textwidth]{./fig_slides/Proper_motion}
        \end{center}
      \end{itemize}
    \end{minipage}
    \only<1>{
      \begin{minipage}[hb]{0.25\linewidth}
        \vspace*{-45pt}
        \hspace*{-5pt}\phantom{\includegraphics[width=1.2\textwidth]{./fig_slides/line2}}
      \end{minipage}
    }
    \only<2->{
      \begin{minipage}[hb]{0.25\linewidth}
        \vspace*{-45pt}
        \hspace*{-5pt}\includegraphics[width=1.2\textwidth]{./fig_slides/line2}
      \end{minipage}}
    \only<2>{
      \begin{textblock}{0.25}[0.5,1](0.5,0.99)
        \centering
        \includegraphics[width=\textwidth]{dark_side_of_the_moon.png}\\
        \textcolor{gray!50}{\tiny Credits: Pink Floyd}
      \end{textblock}
    }
\end{center}
\end{frame}


\bgroup
% \metroset{progressbar=none}
\setbeamertemplate{background}{\centering\includegraphics[width=\paperwidth, keepaspectratio]{sky_empty}}
\begin{frame}
  \frametitle{Measuring distances is one of the most difficult things
    in astrophysics}
  \note{\todo{Explain Parallax!}}
\end{frame}

\egroup
\bgroup
% \metroset{progressbar=none}
\setbeamertemplate{background}{\centering \includegraphics[width=\paperwidth, keepaspectratio]{sky_plane}}
\begin{frame}
  \frametitle{Measuring distances is one of the most difficult things
    in astrophysics}

\end{frame}
\egroup
\bgroup
% \metroset{progressbar=none}
\setbeamertemplate{background}{\centering \includegraphics[width=\paperwidth, keepaspectratio]{sky_plane_fly}}
\begin{frame}
  \frametitle{Measuring distances is one of the most difficult things
    in astrophysics}

\end{frame}
\egroup
\bgroup
% \metroset{progressbar=none}
\setbeamertemplate{background}{\centering \includegraphics[width=\paperwidth, keepaspectratio]{sky_vel}}
\begin{frame}
  \frametitle{Angular velocity + parallax = projected physical velocity}

  \begin{textblock}{0.33}[0.5,0.5](0.75,0.75)
    \includegraphics[width=\textwidth]{Proper_motion}
  \end{textblock}


\end{frame}
\egroup




\section{\textcolor{Blue}{A ``renaissance'' of stellar physics}}
\subsection{driven by the \textcolor{Blue}{\emph{Gaia}} space-telescope}

\begin{frame}{\emph{Gaia}~scans the sky to measure $d$ and
    $v$ of \underline{10s of billions} of stars}
  \note{European Space Agency mission, public data}


  \begin{textblock}{1}[0.5,0.5](0.5,0.15)
    \centering\small
    \url{https://flathub.flatironinstitute.org/gaiadr3}
  \end{textblock}


  \begin{textblock}{0.35}[0,0.5](0.01,0.3)
    \includegraphics[width=100pt]{fig_slides/Gaia_spacecraft_cut_out}
  \end{textblock}
  \begin{textblock}{0.35}[0,1](0.03,0.55)
    \includegraphics[width=40pt]{fig_slides/Gaialogo}
  \end{textblock}

  \begin{textblock}{0.45}[0,0.](0.01,0.56)
    \begin{itemize}\color{black}
    \item Parallax$\equiv$ distance
    \item Velocities ($\mu$ and RV)
    \end{itemize}
    \vspace*{-10pt}
    % \hspace*{40pt}for $\gtrsim 10^9$ stars\\
    \hspace*{70pt}\begin{sideways}
      $\Leftarrow$
    \end{sideways}

    \textbf{\textcolor{Blue}{Enables lots of astrophysics}}
  \end{textblock}

  \begin{textblock}{0.6}[0.5,1](0.66,0.95)
    \centering
    \movie[width=0.9\textwidth,
    height=0.75\textheight,showcontrols=true,autostart=true]{}{./fig_slides/Gaia_scanning_law_no_audio.mp4}\\
    \textcolor{gray!50}{\tiny Credits: New Scientist/ESA}\\[-7pt]
    \textcolor{gray!50}{\tiny \url{https://www.youtube.com/watch?v=BT0Xh1BizSI}}
  \end{textblock}
\end{frame}

\begin{frame}[c,plain]
  \begin{textblock}{0.5}[0,0](0.01,0.01)
    \centering
    \textcolor{Blue}{{\large Apparent motion}\\
      {\small reflection of Earth's orbit around the Sun}}\\[70pt]
    \vfill
      \movie[width=0.9\textwidth,
      height=0.5\textheight,showcontrols=true,autostart=true]{}{./fig_slides/Gaia_parallax.mp4}\\
      \textcolor{gray!50}{\tiny \url{https://www.youtube.com/watch?v=0-jhyRIupY4}}
    \end{textblock}
    \begin{textblock}{0.5}[0,0](0.5,0.01)
      \centering
      \textcolor{Blue}{{\large Long term drift}\\
        {\small Stars' orbit around Galaxy + intrinsic motion}}\\[68pt]
      \vfill
      \movie[width=0.9\textwidth,
      height=0.5\textheight,showcontrols=true,autostart=false]{}{./fig_slides/Gaia_motion.mp4}\\
      \textcolor{gray!50}{\tiny \url{https://www.youtube.com/watch?v=cEsfqFDSpm0}}
  \end{textblock}
\end{frame}



\section{\textcolor{Blue}{Why some star ``run''?}}
\note{By this I mean their ``peculiar'' velocity with respect to the
  Galactic rotation curve is high (say $>30\,\mathrm{km\ s^{-1}}$)}
\subsection{\textcolor{Blue}{\textbullet}~Binary SuperNova scenario
  (BSN) \\\textcolor{Blue}{\textbullet}~Dynamical Ejection Scenario (DES)}


\begin{frame}
  \frametitle{\textcolor{red}{BSN:} the most common massive binary evolution path}
  \centering
  \vspace*{8pt}

  %% multimedia -- okular and pympress
  \movie[width=0.9\textwidth,
  height=0.95\textheight,showcontrols=true,autostart=true]{}{./fig_slides/binary2.mp4}

  \begin{textblock}{1}[0.5,0](0.5,0.13)
    \textcolor{gray!50}{\tiny Credits: ESO, L. Calçada, M. Kornmesser,
      S.E. de Mink  --- \url{https://www.youtube.com/watch?v=pDDjEkGjV9U}}
  \end{textblock}
\end{frame}



\bgroup
\metroset{progressbar=none}
\setbeamertemplate{background}{\includegraphics[width=\paperwidth,height=\paperheight,keepaspectratio]{binary}}
\begin{frame}[plain]
  \note{\todo{modeling is very challenging}}
  % \only<1>{\frametitle{\textcolor{white!80!Yellow}{The most common massive binary evolution path}}}

  % \begin{textblock}{1}[0.5,0.](0.5,0.1)
  %   \centering
  %   \includegraphics[width=\textwidth]{binary}
  % \end{textblock}

  % \only<2>{
  \frametitle{\textcolor{white!80!Yellow}{Mass
      transfer occurs before the  $1^\mathrm{st}$ explosion}}
  \begin{textblock}{0.54}[0.,0.5](0.0,0.28)
    \centering
    \begin{itemize}
    \item[\textcolor{white!80!Yellow}{\textbullet}]
      \textcolor{white!80!Yellow}{\bf Spin-up}\\[-4pt]
      \textcolor{gray!50}{\tiny Packet 1981, Cantiello \emph{et al.}
        2007, de Mink \emph{et al.} 2013, Renzo \& G\"otberg 2021}\\[3pt]
    \item[\textcolor{white!80!Yellow}{\textbullet}]
      \textcolor{white!80!Yellow}{\bf Pollution}\\[-4pt]
      \textcolor{gray!50}{\tiny Blaauw 1993, Renzo \& G\"otberg 2021}\\[3pt]
    \item[\textcolor{white!80!Yellow}{\textbullet}]
      \textcolor{white!80!Yellow}{\bf Rejuvenation}\\[-4pt]
      \textcolor{gray!50}{\tiny Hellings 1983, 1985, Renzo \emph{et al.} 2023}
    \end{itemize}
  \end{textblock}
  % }
  \begin{textblock}{0.5}[0.45,0.5](0.25,0.85)
    \centering
    \textcolor{whiteish}{
      \bf The ``widowed'' star carries signatures of its past in a binary}\\[-2pt]
    \hfill\textcolor{gray!50}{\tiny Renzo \& Zapartas 2020}\ \,
  \end{textblock}
\end{frame}
\egroup


\begin{frame}
  \frametitle{What exactly disrupts  the binary?}


  \only<1>{
    \begin{textblock}{1}[0.5,1](0.5,0.99)
      \centering
      \includegraphics[height=0.85\textheight]{SN_bin.pdf}
    \end{textblock}
  }

\only<2>{
    \begin{textblock}{1}[0.5,1](0.5,0.99)
      \centering
      \includegraphics[height=0.85\textheight]{SN_bin_arrow.pdf}
    \end{textblock}

    \begin{textblock}{0.31}[1,0.5](0.99,0.58)
      %% semi-transparent box
      \addtobeamertemplate{block
        begin}{\pgfsetfillopacity{0.85}}{\pgfsetfillopacity{1}}
      \setbeamercolor{block body}{use=structure,bg=white!80!Yellow}
      \setbeamercolor{block title}{use=structure,bg=white!80!Yellow}
      \begin{block}{\centering \textcolor{red}{\bf SN Natal kick}}
        \textcolor{gray!50}{\tiny (Shklovskii 70, Katz 75, Janka 13, 17)}
      \end{block}
    \end{textblock}
   }

  \begin{textblock}{0.28}[0.5,0.5](0.36,0.37)
    %% semi-transparent box
    \addtobeamertemplate{block
      begin}{\pgfsetfillopacity{0.85}}{\pgfsetfillopacity{1}}
    \setbeamercolor{block body}{use=structure,bg=white!80!Yellow}
    \setbeamercolor{block title}{use=structure,bg=white!80!Yellow}
    \begin{block}{\centering \bf Ejecta impact}
      \textcolor{gray!50}{\tiny (Tauris \&
        Takens 98,
        Liu \emph{et al.} 15)}
    \end{block}
  \end{textblock}

  \begin{textblock}{0.28}[0.5,0.5](0.6,0.8)
    %% semi-transparent box
    \addtobeamertemplate{block
      begin}{\pgfsetfillopacity{0.85}}{\pgfsetfillopacity{1}}
    \setbeamercolor{block body}{use=structure,bg=white!80!Yellow}
    \setbeamercolor{block title}{use=structure,bg=white!80!Yellow}
    \begin{block}{\centering \bf Loss of SN ejecta}
      \textcolor{gray!50}{\tiny (Blaauw '61)}
    \end{block}
  \end{textblock}




  \begin{textblock}{1}[0.5,0.08](0.5,0.15)
    \centering\Large \textcolor{Blue}{{\Huge $86^{+11}_{-22}\%$} of massive binaries are
      disrupted}
  \end{textblock}

  \begin{textblock}{0.5}[0, 1](0.01, 0.99) \textcolor{gray!50}{\tiny
      Renzo \emph{et al.} 19b, Kochanek  \emph{et al.} 19,\\[-7pt]
      Eldridge \emph{et al.} 11, De~Donder \emph{et al. 97}}
\end{textblock}

\end{frame}



\begin{frame}[c]
  \frametitle{Why kicks?~Neutron stars are usually faster than their
    progenitor stars}

  \begin{textblock}{1}[0.5,0.5](0.5,0.53)
    \centering
    \includegraphics[width=\textwidth]{guitar_nebula_complete}\\
    {``Guitar nebula'':  $v_{\rm NS}\simeq1000\,\mathrm{km \ s^{-1}}$}
  \end{textblock}

  \begin{textblock}{1}[0,1](0.01,0.99)
    \textcolor{gray!50}{\tiny Cordes \emph{et al.} 1993, Chatterjee
      \& Cordes 2004, de~Vries
      \emph{et al.} 2022, }
  \end{textblock}
  \begin{textblock}{1}[0.5,1](0.5,0.15)
    \centering \textcolor{gray!50}{\tiny \emph{HST}}
  \end{textblock}

  \only<2>{
    \begin{textblock}{0.2}[1,1](0.92,0.99)
      \includegraphics[width=\textwidth]{fig_slides/bow_wave_cargo}
    \end{textblock}
  }

\end{frame}

\bgroup
\metroset{progressbar=none}
\setbeamercolor{background canvas}{bg=black}
\begin{frame}
  \note{What causes these natal kicks responsible for the binary
  disruption?
  From an observational point of view we know they exists because
  we see pulsars moving much faster than their parent O and early B
  type stars. From a theoretical perspective, we think that these
  kicks are due to asymmetries in the explosion dynamics, either in
  the neutrino flux that drives the explosion (although this is
  presently a bit disfavored) or because of the hydrodynamical
  instabilities in the explosion. For instance, here you see an
  entropy rendering of the core of an exploding 15Msun star, but
  for simplicity you can think of the color as the density. As you
  can see, this is not spherically symmetric: here you have a big
  red clump, which if it is denser, can gravitationally pull the
  proto-compact object (for up to a second) and as long as we have
  some ejecta in the other direction to conserve momentum, we can
  accelerate the proto compact object in the direction of the
  densest ejecta. }
  \frametitle{\textcolor{white!80!Yellow}{Asymmetries in the explosion
    cause the ``kick''}}
  \centering
  % \begin{textblock}{1}[0.5,1](0.5,0.13)
  %   \textcolor{white!80!Yellow}{Observationally:~$v_\mathrm{pulsar} \gg v_\mathrm{OB-stars}$}
  % \end{textblock}

  \only<1>{
    \begin{textblock}{0.2}[1,1](0.99,0.35)
      \includegraphics[width=\textwidth]{onion.png}
    \end{textblock}
    \begin{textblock}{0.3}[0.,0](0.0,0.15)
      \begin{itemize}
        \color{whiteish}
      \item[\textcolor{whiteish}{\textbullet}] $\nu$-driven convection
      \item[\textcolor{whiteish}{\textbullet}] rotation
      \item[\textcolor{whiteish}{\textbullet}] hydrodynamical flow
      \end{itemize}
    \end{textblock}
  }
  % \vspace*{30pt}
  % {
  %   \Large \centering
  %   \textcolor{white!80!Yellow}{
  %   Physically:~$\nu$ emission and/or ejecta anisotropies}\linebreak
  %   \phantom{Bla}
  % }
  % \vspace*{10pt}
  \only<1>{\includegraphics[height=0.95\textheight]{SN_asymmetry}} \only<2>{\hspace*{-6pt}\vspace*{-1pt}\includegraphics[height=0.95\textheight]{SN_asymmetry2}}


  \begin{textblock}{1}[1, 1](0.99,0.99)
    \hfill\textcolor{gray!50}{\tiny Credits: S.~Drasco}\,
  \end{textblock}
\end{frame}
\egroup


\begin{frame}{Kicks do not change companion velocity}

    \begin{textblock}{1}[0.5,1](0.5,0.99)
      \centering
      \includegraphics[height=0.85\textheight]{SN_bin_arrow2.pdf}
    \end{textblock}

    \begin{textblock}{0.3}[1,0.5](0.99,0.58)
      %% semi-transparent box
      \addtobeamertemplate{block
        begin}{\pgfsetfillopacity{0.85}}{\pgfsetfillopacity{1}}
      \setbeamercolor{block body}{use=structure,bg=white!80!Yellow}
      \setbeamercolor{block title}{use=structure,bg=white!80!Yellow}
      \begin{block}{\centering \textcolor{red}{\bf SN Natal kick}}
        \textcolor{gray!50}{\tiny (Shklovskii 70, Katz 75, Janka 13, 17)}
      \end{block}
    \end{textblock}

    \begin{textblock}{0.32}[0.5,0.5](0.2,0.42)
            \addtobeamertemplate{block
        begin}{\pgfsetfillopacity{0.85}}{\pgfsetfillopacity{1}}
      \setbeamercolor{block body}{use=structure,bg=white!80!Yellow}
      \setbeamercolor{block title}{use=structure,bg=white!80!Yellow}
      \begin{block}{\centering \Huge \bf $v_\mathrm{dis}\simeq v_2^\mathrm{orb}$}
        \hfill before the SN\,
      \end{block}
    \end{textblock}


  % \begin{textblock}{0.28}[0.5,0.5](0.36,0.37)
  %   %% semi-transparent box
  %   \addtobeamertemplate{block
  %     begin}{\pgfsetfillopacity{0.85}}{\pgfsetfillopacity{1}}
  %   \setbeamercolor{block body}{use=structure,bg=white!80!Yellow}
  %   \setbeamercolor{block title}{use=structure,bg=white!80!Yellow}
  %   \begin{block}{\centering \bf Ejecta impact}
  %     \textcolor{gray!50}{\tiny (Tauris \&
  %       Takens 98,
  %       Liu \emph{et al.} 15, Hirai \emph{et al.} 18)}
  %   \end{block}
  % \end{textblock}

  % \begin{textblock}{0.28}[0.5,0.5](0.6,0.8)
  %   %% semi-transparent box
  %   \addtobeamertemplate{block
  %     begin}{\pgfsetfillopacity{0.85}}{\pgfsetfillopacity{1}}
  %   \setbeamercolor{block body}{use=structure,bg=white!80!Yellow}
  %   \setbeamercolor{block title}{use=structure,bg=white!80!Yellow}
  %   \begin{block}{\centering \bf Loss of SN ejecta}
  %     \textcolor{gray!50}{\tiny (Blaauw '61)}
  %   \end{block}
  % \end{textblock}

  \begin{textblock}{1}[0.5,0.08](0.5,0.15)
    \centering\Large \textcolor{Blue}{{\Huge $86^{+11}_{-22}\%$} of massive binaries are
      disrupted}
  \end{textblock}

  \begin{textblock}{0.5}[0, 1](0.01, 0.99)
    \textcolor{gray!50}{\tiny Renzo \emph{et al.} 19b, Kochanek
      \emph{et al.} 19,\\[-7pt]
      Eldridge \emph{et al.} 11,
    De~Donder \emph{et al. 97}}
\end{textblock}
\end{frame}




\begin{frame}
  \note{The first thing I want to draw your attention on is of course the velocity
    distribution of ejected widowed stars. Here you see the tail of
    the distribution above 30km/s (the typical threshold to define
    runaways). The three colors correspond to three lower mass cuts.
    The main thing to notice here is that *BINARIES HAVE A VERY HARD
    TIME PRODUCING MASSIVE RUNAWAYS FASTER THAN ~60kms*.}
  \frametitle{Accretor stars can be \emph{runaways}...}
  \centering
  \only<1>{\includegraphics[width=\textwidth]{incomplete_vdist}}

  \begin{textblock}{0.45}[0,1](0.01,0.99)
    \textcolor{gray!50}{\tiny Renzo \emph{et al.} 2019b}\hfill\,
  \end{textblock}

  \begin{textblock}{1}[0.5,0.5](0.5,0.94)
    \textcolor{gray!50}{Velocity w.r.t. pre-explosion binary center of mass}
  \end{textblock}

  \begin{textblock}{1}[0.5,1](0.5,0.99)
    \centering
    \textcolor{gray!50}{\tiny Numerical results:   \href{http://cdsarc.u-strasbg.fr/viz-bin/qcat?J/A+A/624/A66}{http://cdsarc.u-strasbg.fr/viz-bin/qcat?J/A+A/624/A66}}
  \end{textblock}
\end{frame}

\begin{frame}
  \note{Take home points:
    \begin{itemize}
    \item {Walkaways outnumber the runaways by
        $\sim10\times$}
    \item {Binaries barely produce fast runaways}
    \item {All runaways from binaries are post-interaction objects}
    \end{itemize}
  }
  \frametitle{...but most are only \emph{walkaways}}
  \centering
  \includegraphics[width=\textwidth]{complete_vdist}
  \begin{textblock}{0.45}[0,1](0.01,0.99)
    \textcolor{gray!50}{\tiny Renzo \emph{et al.} 2019b}\hfill\,
  \end{textblock}

  \begin{textblock}{1}[0.5,0.5](0.5,0.94)
    \textcolor{gray!50}{Velocity w.r.t. pre-explosion binary center of mass}
  \end{textblock}



  \begin{textblock}{1}[0.5,1](0.5,0.99)
    \centering
    \textcolor{gray!50}{\tiny Numerical results:
      \href{http://cdsarc.u-strasbg.fr/viz-bin/qcat?J/A+A/624/A66}{http://cdsarc.u-strasbg.fr/viz-bin/qcat?J/A+A/624/A66}}
  \end{textblock}




  \only<2>{
    \begin{textblock}{0.95}[0.5,0.5](0.5,0.54)
      \includegraphics[height=0.95\textheight]{binary_arrows}
    \end{textblock}
    \begin{textblock}{0.75}[0.5,0.5](0.5,0.195)
      \textcolor{white!80!Yellow}{\Large Under-production of runaways because}
    \end{textblock}

    \begin{textblock}{0.75}[0.5,0.5](0.5,0.81)
      \textcolor{white!80!Yellow}{\Large mass transfer increases the\linebreak
        accretors' masses\linebreak and \emph{overall} widens the
        binaries}
    \end{textblock}
  }
\end{frame}





\bgroup
\metroset{progressbar=none}
\setbeamertemplate{background}{\includegraphics[width=\paperwidth,height=\paperheight,keepaspectratio]{zoph_onethird}}
\begin{frame}
  \frametitle{\textcolor{whiteish}{Constrain BSN with the nearest massive star to
      Earth}}

  \begin{textblock}{1}[0,1](0.34,0.99)
    \textcolor{gray!50}{\tiny
      Walker \emph{et al.} 1979,\\
      Herrero \emph{et al.} 1994,\\
      van~Rensbergen \emph{et al.} 1996,\\
      Hoogerwerf \emph{et al.} 2001,\\
      Villamariz \& Herrero 2005,\\
      Walker \& Koushnik 2005,\\
      Zee \emph{et al.} 2018,\\
      Gordon \emph{et al.} 2018,\\
      Neuh\"auser \emph{et al.} 2019, 2020,\\
      Renzo \& G\"otberg 2021,\\[-7pt]
      Shepard \emph{et al.} 2022
    }
  \end{textblock}

  \begin{textblock}{0.33}[0,1](0.01,0.99)
    \textcolor{gray!50}{\tiny Spitzer, NASA/JPL}
  \end{textblock}


  % \only<2>{
  %   \begin{textblock}{0.45}[1,0.5](0.98,0.5)
  %     \centering
  %     \includegraphics[width=0.75\textwidth]{bow_wave_cargo}\\
  %     % \end{textblock}
  %     % \begin{textblock}{0.4}[1,0](0.895,0.79)
  %     \textcolor{gray!50}{\tiny e.g., \underline{Sexton \emph{et al.} 2015, Kiminki \emph{et al.} 2017},\\[-7pt]
  %       Bodensteiner \emph{et al.} 2018, Raga \emph{et al.} 2022}\ \,
  %   \end{textblock}
  % }



  \only<2>{
    \begin{textblock}{0.45}[1,0.5](0.98, 0.52)
      \begin{block}{Observational constraints of $\zeta$ Oph.:}
        \begin{itemize}
        \item distance $\simeq 107\pm4\,\mathrm{pc}$
        \item total mass$\simeq 20\,M_\odot$
        \item speed $\gtrsim 30\,\mathrm{km\ s^{-1}}$
        \item rotation $v\sin(i)\gtrsim 310$
        \item inclination $i \gtrsim 56^\circ$
        \item ``looks'' $(T_\mathrm{eff}, L)$
        \item age $\simeq$ 8\,Myr
        \item strange surface composition
        \item[\textcolor{red}{\xmark}] \textcolor{red}{\bf Rotating
            single stars}\\
          \hspace*{-23pt}\textcolor{gray!50}{\tiny (e.g., van Rensbergen \emph{et
              al.} 96, Howarth \& Smith 01, Villamariz \& Herrero 05)} \,

        \end{itemize}
      \end{block}
    \end{textblock}
  }
\end{frame}


\begin{frame}
  \frametitle{\textcolor{whiteish}{$\zeta$ Ophiuchi is single but we can trace it back to a neutron star}}

  \begin{textblock}{0.5}[0.5,0](0.75,0.12)
    \includegraphics[height=.9\textheight]{kinematics_zeta.pdf}
  \end{textblock}


  % \begin{textblock}{0.6}[1,0.5](1,0.5)
  %   \centering
  %   \includegraphics[width=\textwidth]{neuhauser20_title}
  % \end{textblock}


  \begin{textblock}{1}[0,1](0.34,0.99)
    \textcolor{gray!50}{\tiny Neuh\"auser \emph{et al.} 2019, 2020 see
      also Blaauw 1952, 1961,\\[-7pt]
      van Rensbergen \emph{et al.} 1996,
      Hoogerwerf \emph{et al.} 2001, Lux \emph{et al.} 2020}\hfill\,
  \end{textblock}

  \only<2>{
    \begin{textblock}{0.4}[0.5,0.5](0.68,0.81)
      \centering\small
      \begin{block}{\centering SN explosion $\sim$$1.78\pm0.21$\,Myr ago\\}
        \centering
        $\Rightarrow$Radioactive iron rain on Earth\\
        \hfill\textcolor{gray!50}{\tiny Benitez \emph{et al.} 2002, Fry \emph{et al.} 2016, Neuh\"auser \emph{et al.} 2020}
      \end{block}
    \end{textblock}

  }
\end{frame}
\egroup



\section{\textcolor{Blue}{Why some star ``run''?}}
\subsection{\textcolor{gray!40}{\textbullet~Binary SuperNova scenario
  (BSN)} \\\textcolor{Blue}{\textbullet}~Dynamical Ejection Scenario (DES)}

\bgroup
\metroset{progressbar=none}
\setbeamertemplate{background}{\centering \includegraphics[width=\paperwidth, keepaspectratio]{JWST_R136}}
\begin{frame}
  \note{Emphasize $~10^{5}$ stars in the volume between the Sun and
    Proxima centauri:
    \begin{itemize}
    \item $\sim10^5$ stars within $\sim 1\,$pc
    \item current age $\lesssim 2$\,Myr
    \item distance $\sim 50\,$ kpc
    \end{itemize}
  }
  \begin{textblock}{0.55}[0,1](0.01,0.1)
  \begin{tcolorbox}[boxrule=0pt,standard jigsaw, colback=black,
      left=0pt, right=0pt, top=0pt, bottom=0pt,
      colframe=black,
      opacityback=0.5]
      \textcolor{whiteish}{\large \bf Most stars are born in dense environments}
  \end{tcolorbox}
  \end{textblock}
  \only<2>{
    \begin{textblock}{0.4}[1., 0.5](0.99, 0.35)
      \begin{block}{\centering N-body interactions}
         (typically) least massive thrown out.\linebreak
         \phantom{AAAAAA}\textcolor{red}{Binaries matter...}
          \begin{itemize}
          \item Cross section $\propto a^2 \gg R_*^2$
          \item (Binding) Energy reservoir \\
            \textcolor{gray!50}{\tiny Poveda \emph{et al.} 67}
          \end{itemize}
          \textcolor{red}{...but don't necessarily leave imprints!}
        \end{block}
    \end{textblock}
}

  \begin{textblock}{1}[0,1](0.01,0.99)
    \textcolor{gray!50}{\tiny \emph{JWST} view of R136 in the LMC}
  \end{textblock}

\end{frame}
\egroup

\bgroup
\metroset{progressbar=none}
\setbeamercolor{background canvas}{bg=black}
\begin{frame}{\textcolor{red}{DES:}~\textcolor{whiteish}{typically eject the lowest mass star}}
  \centering
  \movie[width=0.9\textwidth,
  height=0.95\textheight,showcontrols=true,autostart=true]{}{./fig_slides/nopn_dvd.mp4}

  \begin{textblock}{1}[0.,1](0.01,0.99)
    \textcolor{gray!50}{\tiny Credits: Carl Rodriguez --- \url{https://dynamics.unc.edu/public-outreach/outreach-movies/}}
  \end{textblock}

\end{frame}



\begin{frame}
   \frametitle{\textcolor{white!80!Yellow}{Typical outcome of dynamical
       interactions}}
   \begin{textblock}{1}[0.5,0.5](0.5,0.55)
     \centering
     \includegraphics[width=0.8\textwidth]{fig_slides/dyn_res}
   \end{textblock}

   \begin{textblock}{0.5}[0.5,0.01](0.65,0.18)
     \centering
     \textcolor{white!80!Yellow}{\Large \bf Fast runaway }
   \end{textblock}


   \begin{textblock}{0.9}[0.5,0.5](0.5,0.9)
     \centering
     \textcolor{white!80!Yellow}{\Large \bf \phantom{Fast ejection +}Tighter and more massive binary}\\
     \hfill\textcolor{gray!50}{\tiny e.g., Fujii \& Portegies-Zwart
       11}
   \end{textblock}


\end{frame}
\egroup


\begin{frame}
  \frametitle{Timing of ejection}

  \includegraphics[height=0.9\textheight]{fig_slides/oh_kroupa16}
  %\textcolor{gray!50}{\tiny from Oh \& Kroupa 16}

  \begin{textblock}{1}[0,1](0.01,0.99)
    \textcolor{gray!50}{\tiny from Oh \& Kroupa 16, \linebreak see also, Poveda \emph{et
      al.} 64, Fujii \& Portegies-Zwart 11, Banerjee \emph{et al.} 12,
  14}\hfill\,
  \end{textblock}

  \begin{textblock}{0.45}[0.5,0.5](0.75,0.5)
    \centering
    \textcolor{Blue}{Most ejections happen early}\\
    \textcolor{Black}{Before the first stellar death}\\
    $$ \tau_\mathrm{ej} < \tau_*$$\\
    \vspace*{30pt}
    \textcolor{Blue}{{\bf Very} sensitive to initial conditions}
  \end{textblock}
\end{frame}


\begin{frame}
  \frametitle{The most massive runaways known}
  \vspace*{5pt}
  \only<1>{\includegraphics[height=0.95\textheight]{fig_slides/R136_clean.pdf}}
  \only<2>{\includegraphics[height=0.95\textheight]{fig_slides/682_gaia_only}}
  \only<3>{\includegraphics[height=0.95\textheight]{fig_slides/682_gaia_hst_only}}
  \only<4>{\includegraphics[height=0.95\textheight]{fig_slides/682_complete}}
  \begin{textblock}{1}[0.5,1](0.5,0.99)
    \textcolor{gray!50}{\tiny \phantom{A}Renzo \emph{et al.} 2019a\hfill
      Lennon \emph{et al.}, 2018\ \,}
  \end{textblock}

    \begin{textblock}{0.18}[0.,0](0.7,0.65)
      \setbeamercolor{block title}{use=structure,fg=Blue,bg=white!80!Yellow}
      \setbeamercolor{block body}{use=structure,fg=black,bg=white!80!Yellow}
      \begin{block}{}
        \scriptsize
        \hspace*{5pt}\vspace*{3pt}$M=91.6^{+11.5}_{-10.5}\,M_\odot$\\
        \only<3-4>{\hspace*{5pt}\vspace*{3pt}$v_\mathrm{2D}=80\pm11\,\mathrm{km\ s^{-1}}$\\
        % \hspace*{5pt}\vspace*{3pt}$v_\mathrm{3D}=112\pm8\,\mathrm{km\
        % s^{-1}}$
        }
      \end{block}
    \end{textblock}


    \begin{textblock}{0.18}[0,0](0.7,0.25)
      \setbeamercolor{block title}{use=structure,fg=Blue,bg=white!80!Yellow}
      \setbeamercolor{block body}{use=structure,fg=black,bg=white!80!Yellow}
      \begin{block}{}
        \scriptsize
        \hspace*{5pt}\vspace*{3pt}$M=97.6^{+22.2}_{-23.1}\,M_\odot$\\
        % $\mathrm{age} =0.4^{+0.8}_{-0.4}$\,Myr\\
        % $\tau_\mathrm{kin}=0.9\pm0.15$\,Myr\\
        % \only<1-2>{\hspace*{5pt}\vspace*{3pt}$v_\mathrm{LOS}\simeq0\,\mathrm{km\ s^{-1}}$}
        \only<3-4>{\hspace*{5pt}\vspace*{3pt}$v_\mathrm{2D}=93\pm15\,\mathrm{km\
          s^{-1}}$\\
        % \hspace*{5pt}\vspace*{3pt}$v_\mathrm{3D}\simeq v_\mathrm{2D}$
      }
      \end{block}
    \end{textblock}

    \begin{textblock}{0.18}[1,0](0.3,0.13)
      \setbeamercolor{block title}{use=structure,fg=Blue,bg=white!80!Yellow}
      \setbeamercolor{block body}{use=structure,fg=black,bg=white!80!Yellow}
      \begin{block}{}
        \scriptsize
        \hspace*{5pt}\vspace*{5pt}$M =
        137.8^{+27.5}_{-15.9}\,M_\odot$\\
        \only<4>{\hspace*{5pt}\vspace*{5pt}$v_\mathrm{2D}=38\pm17\,\mathrm{km\
          s^{-1}}$}
      \end{block}
    \end{textblock}
\end{frame}


\begin{frame}
  \frametitle{Summary of ejection mechanisms}
   \begin{columns}
      \begin{column}{.48\textwidth}
      \begin{textblock}{0.45}[0,0](0.01,0.12)
        \begin{block}{\centering \bf Dynamical ejections}
          \begin{itemize}
          \item Happen before SNe
          \item Can produce higher velocities %($v\gtrsim100\,\mathrm{km\ s^{-1}}$)
          \item Least massive thrown out
          \item \emph{Gaia} hint: high efficiency
          \end{itemize}
          \textcolor{red}{\small ...Binaries are still important!}
          % \begin{itemize}
          % \item (Binding) Energy reservoir
          % \item Cross section $\propto a^2 \gg R_*^2$
          % \end{itemize}
          \textcolor{red}{\small but might not leave signature}
          \begin{center}
            \begin{textblock}{0.45}[0,0](0.01,0.658)
              \includegraphics[width=170pt]{JWST_R136}\\
            \end{textblock}
            \end{center}
        \end{block}
      \end{textblock}
    \end{column}
    \begin{column}{.48\textwidth}
       \begin{textblock}{0.45}[1,0](0.99,0.12)
         \begin{block}{\centering \bf Binary SN disruption}
           \begin{itemize}
           \item Most binaries are disrupted
           \item Determined by SN kick
           \item Ejects accretor
           \item $v \simeq v_2^\mathrm{orb}$ typically slow
           \item Leaves \textcolor{red}{binary signature} \\[3pt]
             {\small \textcolor{red}{spin up, pollution, rejuvenation}}
           \end{itemize}
           \begin{textblock}{0.45}[1,0](0.999,0.658)
             \includegraphics[width=170pt]{complete.png}\\
           \end{textblock}
         \end{block}
       \end{textblock}
     \end{column}
   \end{columns}

   \only<2>{
     \begin{textblock}{0.4}[0.5,0.5](0.5,0.8)
       \setbeamercolor{block title}{use=structure,fg=Blue,bg=whiteish}
       \setbeamercolor{block body}{use=structure,fg=black,bg=whiteish}
       \begin{block}{\centering Relative efficiency \textcolor{red}{\huge ?}}
       % \centering
       % $\sim\,10\%$ O-type stars are runaways\\
       % \vspace*{1em}
        % $\sim$\,$\frac{2}{3}$ of runaways from binaries\\
       \hfill\textcolor{gray!50}{\tiny Hoogerwerf \emph{et al.} 2001,
         Jilinksi \emph{et al.} 2010, Sana \emph{et al.} 2023}\,
       \end{block}
     \end{textblock}
   }

\end{frame}


\begin{frame}{\emph{Both} mechanism at play in the same population}


  \begin{textblock}{0.3}[0,0](0.01,0.15)
    \centering
    \includegraphics[width=\textwidth]{ORWsLMC}\\[-10pt]
    \textcolor{gray!50}{\tiny\phantom{AAAA}(30 Doradus region of LMC)}
  \end{textblock}

  \begin{textblock}{0.65}[0.5,0.5](0.75,0.55)
    \only<1-2>{\includegraphics[width=0.8\textwidth]{vsini_rot}}
    \only<3>{\includegraphics[width=0.8\textwidth]{vsini_3D}}
  \end{textblock}

  \begin{textblock}{0.1}[0.5,0.5](0.43,0.5)
    \begin{sideways}
      \bf Projected rotational velocity
    \end{sideways}
  \end{textblock}

  \begin{textblock}{0.65}[0.5,0.5](0.95,0.95)
    \only<1-2>{\bf Line-of-sight velocity}
    \only<3>{\bf 3D velocity}
  \end{textblock}

  \only<2>{
    \begin{textblock}{0.11}[0.,0.](0.5,0.16)
      \begin{tcolorbox}[boxrule=2pt,standard jigsaw, colback=whiteish,
      left=0pt, right=0pt, top=0pt, bottom=0pt,
      colframe=red,
      opacityback=1]
      \centering
      \textcolor{red}{\huge BSN}
      \end{tcolorbox}
    \end{textblock}

    \begin{textblock}{0.11}[0.,0.5](0.85,0.85)
      \begin{tcolorbox}[boxrule=2pt,standard jigsaw, colback=whiteish,
      left=0pt, right=0pt, top=0pt, bottom=0pt,
      colframe=blue,
      opacityback=1]
      \centering
      \textcolor{blue}{\huge DES}
      \end{tcolorbox}
    \end{textblock}
  }
  \only<2-3>{
    \begin{textblock}{0.3}[0,0](0.01,0.72)
      \centering
      General conclusion\\
      or\\
      lucky age of 30\,Doradus
      {\huge \textcolor{red}{?}}
    \end{textblock}
  }



  \begin{textblock}{1}[0,1](0.01,0.99)
    \textcolor{gray!50}{\tiny Sana \emph{et al.} 2023}
  \end{textblock}

\end{frame}

\section{\textcolor{Blue}{Conclusions}}
\subsection{}



\begin{frame}
  \frametitle{\emph{Gaia} ``renaissance'' of stellar
    physics $\Rightarrow$ \textcolor{red}{wide} implications for astro}

  \begin{textblock}{0.3}[0,0.5](0.01,0.5)
    \centering
  \movie[width=\textwidth,height=0.5\textheight,
  showcontrols=true,autostart=true]{}{./fig_slides/what_can_we_learn.mp4}\\
  \textcolor{gray!50}{\tiny \url{www.youtube.com/watch?v=jKWQmbB5EQE}}
  \end{textblock}


  \begin{textblock}{0.62}[1,0.5](0.97,0.5)
    \begin{block}{Runaways: \textcolor{black}{\normalfont
          many massive stars are fast $v\gtrsim 30\,\mathrm{km\ s^{-1}}$}}
      Two mechanisms for $v\lesssim$ hundreds of $\mathrm{km\ s^{-1}}$:
      \begin{itemize}
      \item \textcolor{Blue}{\textbf{BSN}} Supernova explosion in a binary
      \item \textcolor{Blue}{\textbf{DES}} Dynamical ejection from clusters
      \end{itemize}
    \end{block}
  \end{textblock}

\end{frame}

\bgroup
\setbeamercolor{background canvas}{bg=black}
\begin{frame}[plain]
  \note{  To conclude, I want to show you a widowed star that you can find
    in the skies of the Netherlands with your naked eyes. Some of
    you probably recognize this constellation, it's orion. It is
    visible in the winter, and you can recognize it from the "orion
    belt". This dongle here is full of massive stars. But here, you
    see Betelgeuse, a famous massive star because it is said to be
    about to explode (i.e. in the next ~1000 years). The interesting
    thing, is that this is also most likely a widowed star: it moves
    at 30km/s and also has the bow shock (like the wave in front of
    the boat!). So every night, you can see one of the stars that I
    wish we will use to understand BH formation and SN explosion
    without seeing either the BH or the SN.
  }
  \begin{textblock}{1}[0.5,0.5](0.5,0.5)
    \centering
    \includegraphics[height=\paperheight]{fig_slides/orion}
  \end{textblock}


  \begin{textblock}{0.04}[0.5,0.5](0.97,0.5)
      \begin{block}{}
            \begin{sideways}
              \color{Blue}
              \centering
              \Large Orion
            \end{sideways}
          \end{block}
  \end{textblock}
\end{frame}

\begin{frame}[plain]
  \begin{textblock}{1}[0.5,0.5](0.5,0.5)
    \centering
    \includegraphics[height=\paperheight]{fig_slides/orion2}
  \end{textblock}

  \begin{textblock}{0.04}[0.5,0.5](0.97,0.5)
      \begin{block}{}
            \begin{sideways}
              \color{Blue}
              \centering
              \Large Orion
            \end{sideways}
          \end{block}
  \end{textblock}
\end{frame}


\begin{frame}[plain]
  \begin{textblock}{1}[0.5,0.5](0.5,0.5)
    \centering
    \includegraphics[height=\paperheight]{fig_slides/orion3}
  \end{textblock}



  \begin{textblock}{0.04}[0.5,0.5](0.97,0.5)
      \begin{block}{}
            \begin{sideways}
              \color{Blue}
              \centering
              \Large Orion
            \end{sideways}
          \end{block}
  \end{textblock}
\end{frame}
\egroup




\appendix

\bgroup
\metroset{progressbar=none}
\setbeamertemplate{background}{\includegraphics[width=\paperwidth,height=\paperheight,keepaspectratio]{bin_expl_shade.png}}
\begin{frame}[plain]% {\textcolor{whiteish}{The explosive connection between binaries and transients}}
  \begin{textblock}{0.14}[0.5,0.5](0.666,0.45)
    % \includegraphics[width=\textwidth]{link_chain_blue.png}
    \begin{block}{}
      \centering
      \textcolor{Blue}{\Large \bf Backup}
    \end{block}
  \end{textblock}
\end{frame}
\egroup

\begin{frame}[c]
  \frametitle{Do BHs receive kicks\textcolor{red}{\bf ?}}
  \vspace*{5pt}
  \begin{columns}
    \begin{column}{0.5\textwidth}
      \centering
      {\Huge NO}\\
      {$\Rightarrow$ most remain bound to companion}\\[20pt]
      \includegraphics[width=120pt]{fig_slides/XRB}
    \end{column}
    \begin{column}{0.5\textwidth}
      \centering
      {\Huge YES}\\
      {$\Rightarrow$ most are single and we can't see them...}\\[20pt]
      \includegraphics[width=120pt]{fig_slides/BH}
  \end{column}
\end{columns}

\only<2>{
  \begin{textblock}{0.4}[0.50,0](0.75,0.86)
    \centering
    \textcolor{Blue}{\Large ...but we can see the \mbox{``widowed'' companions}}
  \end{textblock}
}

\end{frame}


\begin{frame}
  \frametitle{BH kicks by weighting the companions}
  \centering
  %\vspace*{18pt}
  % \includegraphics[height=0.9\textheight]{fig_slides/BH_MF_sigma100}
  \only<1>{
    \begin{textblock}{1}[0.5,0.5](0.5,0.525)
      \includegraphics[height=0.8\textheight]{fig_slides/BH_MFempty}
    \end{textblock}
  }
  \only<2>{
    \begin{textblock}{1}[0.5,0.5](0.5,0.525)
      \includegraphics[height=0.8\textheight]{fig_slides/BH_MF1}
    \end{textblock}
    }
    \only<3>{
      \begin{textblock}{1}[0.5,0.5](0.5,0.525)
        \includegraphics[height=0.8\textheight]{fig_slides/BH_MF2}
      \end{textblock}
  }
  \only<4>{
    \begin{textblock}{1}[0.5,0.5](0.5,0.525)
      \includegraphics[height=0.8\textheight]{fig_slides/BH_MFcomplete}
      \end{textblock}
    }
    \begin{textblock}{1}[0.5,1](0.52,0.92)
      \bf Mass
    \end{textblock}
    \begin{textblock}{0.065}[0,0.5](0.1,0.5)
      \begin{sideways}
        \bf \# stars
      \end{sideways}
    \end{textblock}

   % \begin{textblock}{0.25}[1,1](0.99,0.93)
   %    \includegraphics[width=100pt]{fig_slides/Gaia_spacecraft_cut_out}
   %  \end{textblock}
   %  \begin{textblock}{0.25}[1,1](0.97,0.999)
   %    \includegraphics[width=40pt]{fig_slides/Gaialogo}
   %  \end{textblock}


  \begin{textblock}{1}[0.53,0](0.5,0.12)
      \centering
      \small
      {Massive runaways mass function
        ($v\geq30\,\mathrm{km\ s^{-1}}$, $M\geq7.5\,M_\odot$)}
  \end{textblock}
  \begin{textblock}{0.45}[0,1](0.01,0.99)
    \textcolor{gray!50}{\tiny Renzo \emph{et al.} 19b}\hfill\,
  \end{textblock}

  \begin{textblock}{1}[0.5,1](0.5,0.99)
    \centering
    \textcolor{gray!50}{\tiny Numerical results publicly available at::\\[-5pt]
      \href{http://cdsarc.u-strasbg.fr/viz-bin/qcat?J/A+A/624/A66}{http://cdsarc.u-strasbg.fr/viz-bin/qcat?J/A+A/624/A66}}
  \end{textblock}
\end{frame}

\bgroup
\metroset{progressbar=none}
\setbeamertemplate{background}{\includegraphics[width=\paperwidth,height=\paperheight,keepaspectratio]{binary}}
\begin{frame}{\textcolor{white!80!Yellow}{Self-consistent}~\includegraphics[width=50pt]{mesa}~\textcolor{white!80!Yellow}{model}}
  % \begin{textblock}{1}[0.5,0](0.5,0.1)
  %   \centering
  %   \includegraphics[width=\textwidth]{binary}
  % \end{textblock}


  \begin{textblock}{0.3}[0.5,0.5](0.34,0.46)
    \textcolor{white!80!Yellow}{$M_2=17\,M_\odot$}
  \end{textblock}

  \begin{textblock}{0.3}[0.5,0.5](0.85,0.87)
    \textcolor{white!80!Yellow}{$M_1=25\,M_\odot$}
  \end{textblock}

  \begin{textblock}{1}[0.5,0.5](0.45,0.87)
    \centering
    \textcolor{white!80!Yellow}{$P=100\,\mathrm{days}$}\\[2pt]
    \textcolor{gray!50}{\small (case B RLOF)}
  \end{textblock}

  % \begin{textblock}{1}[1,1](0.99,0.99)
  %   \centering
  %   \hfill\textcolor{gray!50}{\tiny To detach \includegraphics[height=5pt]{mesa}
  %   binary on the fly:
  %   \href{https://github.com/MESAHub/mesa-contrib/}{https://github.com/MESAHub/mesa-contrib/}
  %   $\rightarrow$ \texttt{hooks/detach\_binary}}
  % \end{textblock}


  \begin{textblock}{0.3}[0.5,0.5](0.08,0.3)
    \centering
    \textcolor{white!80!Yellow}{$Z=0.01$}\\
    \textcolor{gray!50}{\tiny (Murphy \emph{et al.} 2021)}
  \end{textblock}

  \begin{textblock}{1}[0,1](0.01,0.99)
    \textcolor{gray!50}{\tiny Renzo \& G\"otberg 2021}\,
  \end{textblock}
\end{frame}
\egroup

\bgroup
\metroset{progressbar=none}
\setbeamertemplate{background}{\includegraphics[width=\paperwidth,height=\paperheight,keepaspectratio]{bin_expl.png}}
\begin{frame}[plain]

  \begin{textblock}{0.33}[0.5,1](0.5,0.99)
    \centering
    \textcolor{gray!50}{\tiny Not simultaneous!}
  \end{textblock}
\end{frame}
\egroup

\bgroup
\metroset{progressbar=none}
\setbeamertemplate{background}{\includegraphics[width=\paperwidth,height=\paperheight,keepaspectratio]{complete.png}}
\begin{frame}[plain]


  \begin{textblock}{0.33}[0.5,1](0.5,0.99)
    \centering
    \textcolor{gray!50}{\tiny Not simultaneous!}
  \end{textblock}
\end{frame}
\egroup


\bgroup
\metroset{progressbar=none}
\setbeamertemplate{background}{\includegraphics[width=\paperwidth,height=\paperheight,keepaspectratio]{zoph_onethird}}
\begin{frame}

  \only<1>{
    \begin{textblock}{0.66}[0,0.5](0.34,0.5)
      \centering
      \textcolor{whiteish}{\Large \bf Does a binary
        past help with $\zeta$ Oph.~{\Huge?}}\\
      \textcolor{whiteish}{\large Spin-up -- Pollution --
        Rejuvenation}\\[-3pt]
      \hfill{\textbf{\textcolor{gray!50}{\tiny Renzo \& G\"otberg 2021}}}\ \,
    \end{textblock}

  }

  \only<2>{
    \begin{textblock}{0.66}[0,0](0.34,0.1)
      \centering
      \textcolor{red}{\Large \xmark}~\textcolor{whiteish}{\Large \bf Spin up:}\\
      \textcolor{whiteish}{\large Natal rotation would need to be extreme
        to match}
    \end{textblock}
    \begin{textblock}{0.66}[0,1](0.34,0.95)
      \centering
      \includegraphics[width=0.75\textwidth]{rot_surf_single_only}\linebreak
      \textcolor{gray!50}{\footnotesize weak-wind problem, neglecting inclination}
    \end{textblock}


    \begin{textblock}{0.66}[0,1](0.34,0.99)
      \textcolor{gray!50}{\tiny Renzo \& G\"otberg 2021}\,
    \end{textblock}
  }
  \only<3>{
    \begin{textblock}{0.66}[0,0](0.34,0.1)
      \centering
      \textcolor{green!80!black}{\Large \cmark}~\textcolor{whiteish}{\Large \bf Spin up:}\\
      \textcolor{whiteish}{\large late and to critical rotation}
    \end{textblock}
    \begin{textblock}{0.66}[0,1](0.34,0.95)
      \centering
      \includegraphics[width=0.75\textwidth]{rot_surf_acc}\linebreak
      \textcolor{gray!50}{\footnotesize weak-wind problem, neglecting inclination}
    \end{textblock}


    \begin{textblock}{0.66}[0,1](0.34,0.99)
      \textcolor{gray!50}{\tiny Renzo \& G\"otberg 2021}\,
    \end{textblock}
  }
  \only<4>{
    \begin{textblock}{0.66}[0,0](0.34,0.1)
      \centering
      \textcolor{green!80!black}{\Large \cmark}~\textcolor{whiteish}{\Large \bf Spin up:}\\
      \textcolor{whiteish}{\large late and to critical rotation}
    \end{textblock}
    \begin{textblock}{0.66}[0,1](0.34,0.95)
      \centering
      \includegraphics[width=0.75\textwidth]{rot_surf_acc_inset}\linebreak
      \textcolor{gray!50}{\footnotesize weak-wind problem, neglecting inclination}
    \end{textblock}


    \begin{textblock}{0.66}[0,1](0.34,0.99)
      \textcolor{gray!50}{\tiny Renzo \& G\"otberg 2021}\,
    \end{textblock}
  }



  \only<5>{
    \begin{textblock}{0.66}[0,0](0.34,0.1)
      \centering
      \textcolor{green!80!black}{\Large \cmark}~\textcolor{whiteish}{\Large \bf Pollution:}\\
      \textcolor{whiteish}{\large Surface composition partly comes
        from the donor's core}
    \end{textblock}
    \begin{textblock}{0.66}[0,0.5](0.34,0.57)
      \centering
      \includegraphics[width=0.5\textwidth]{N14_only.pdf}\linebreak
      \textcolor{gray!50}{\footnotesize Joint constrain on accretion
        and internal mixing}
    \end{textblock}

    \begin{textblock}{0.1}[0.5,0.5](0.91,0.34)
      \begin{rotate}{-90}
        \textcolor{whiteish}{Nitrogen mass fraction}
      \end{rotate}
    \end{textblock}


    \begin{textblock}{0.66}[0,1](0.34,0.99)
      \textcolor{gray!50}{\tiny Renzo \& G\"otberg 2021}\,
    \end{textblock}
  }

  \only<6>{
    \begin{textblock}{0.66}[0,0](0.34,0.1)
      \centering
      \textcolor{green!80!black}{\Large \cmark}~\textcolor{whiteish}{\Large \bf Pollution:}\\
      \textcolor{whiteish}{\large Surface composition partly comes
        from the donor's core}
    \end{textblock}
    \begin{textblock}{0.66}[0,0.5](0.34,0.57)
      \centering
      \includegraphics[width=0.5\textwidth]{mix.pdf}\linebreak
      \textcolor{gray!50}{\footnotesize Joint constrain on accretion
        and internal mixing}
    \end{textblock}

    \begin{textblock}{0.1}[0.5,0.5](0.91,0.34)
      \begin{rotate}{-90}
        \textcolor{whiteish}{Nitrogen mass fraction}
      \end{rotate}
    \end{textblock}


    \begin{textblock}{0.66}[0,1](0.34,0.99)
      \textcolor{gray!50}{\tiny Renzo \& G\"otberg 2021}\,
    \end{textblock}


    \begin{textblock}{0.1}[0.5,0.5](0.51,0.52)
      \begin{sideways}
        \textcolor{whiteish}{``Mixing strength''}
      \end{sideways}
    \end{textblock}


  }

  \only<7>{
    \note{binding energy of the envelope, reverse shock SNe}
    \begin{textblock}{0.66}[0,0](0.34,0.1)
      \centering
      \textcolor{green!80!black}{\Large \cmark}~\textcolor{whiteish}{\Large \bf Rejuvenation:}\\
      \textcolor{whiteish}{\large Core growth changes its outer boundary}
    \end{textblock}
    \begin{textblock}{0.66}[0,0.5](0.34,0.57)
      \centering
      \includegraphics[width=0.5\textwidth]{rho_comparison}\linebreak
      \textcolor{gray!50}{\footnotesize \@ end of H-core burning,\\
        later evolution amplifies differences\\[-5pt]
        {\tiny (e.g., Renzo \emph{et al.} 2017)}}
    \end{textblock}

    \begin{textblock}{0.1}[0.5,0.5](0.51,0.52)
      \begin{sideways}
        \textcolor{whiteish}{``Density''}
      \end{sideways}
    \end{textblock}


    \begin{textblock}{0.66}[0,1](0.34,0.99)
      \textcolor{gray!50}{\tiny Renzo \& G\"otberg 2021, Renzo \emph{et al.} 2023}\,
    \end{textblock}
  }
\end{frame}



\end{document}

%%% Local Variables:
%%% mode: latex
%%% TeX-master: t
%%% End:
